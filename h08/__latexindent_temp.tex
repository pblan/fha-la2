\documentclass[answers]{exam}

\usepackage[ngerman, shorthands=off]{babel}
\usepackage{amsmath,amsthm,amsfonts,stmaryrd,amssymb,mathtools}
\usepackage{xcolor,soul}
\usepackage{polynom}
\usepackage{tikz}
\usetikzlibrary{arrows.meta,angles,quotes,calc}
\usepackage{footnote}
\usepackage{nicefrac}
\usepackage{siunitx}
\usepackage{array}   % for \newcolumntype macro
\usepackage{pgfplots}
\usepgfplotslibrary{fillbetween}
\pgfplotsset{compat=1.8}

\newcolumntype{L}{>{$}l<{$}} % math-mode version of "l" column type
\newcolumntype{R}{>{$}r<{$}} % math-mode version of "r" column type
\newcolumntype{C}{>{$}c<{$}} % math-mode version of "c" column type
\newcolumntype{P}{>{$}p<{$}} % math-mode version of "l" column type

\renewcommand*\env@matrix[1][*\c@MaxMatrixCols c]{%
  \hskip -\arraycolsep
  \let\@ifnextchar\new@ifnextchar
  \array{#1}}

  \newenvironment{sysmatrix}[1]
  {\left(\begin{array}{@{}#1@{}}}
  {\end{array}\right)}

  \newcommand{\Rnum}[1]{\uppercase\expandafter{\romannumeral #1\relax}}
\newcommand{\abs}[1]{\left| #1 \right|}
\newcommand{\cis}[1]{\left( \cos\left( #1 \right) + i \sin\left( #1 \right) \right)}
\newcommand{\sgn}{\text{sgn}} % Signum-Funktion
\newcommand{\diff}{\mathrm{d}} % Differentialquotienten d
\newcommand{\dx}{~\mathrm{d}x} % dx
\newcommand{\du}{~\mathrm{d}u} % du
\newcommand{\dv}{~\mathrm{d}v} % dv
\newcommand{\dw}{~\mathrm{d}w} % dw
\newcommand{\dt}{~\mathrm{d}t} % dt
\newcommand{\dn}{~\mathrm{d}n} % dn
\newcommand{\dudx}{~\frac{\mathrm{d}u}{\mathrm{d}x}} % du/dx
\newcommand{\dudn}{~\frac{\mathrm{d}u}{\mathrm{d}n}} % du/dn
\newcommand{\dvdx}{~\frac{\mathrm{d}v}{\mathrm{d}x}} % dv/dx
\newcommand{\dwdx}{~\frac{\mathrm{d}w}{\mathrm{d}x}} % dw/dx
\newcommand{\dtdx}{~\frac{\mathrm{d}t}{\mathrm{d}x}} % dt/dx
\newcommand{\ddx}{\frac{\mathrm{d}}{\mathrm{d}x}} % d/dx
\newcommand{\dFdx}{\frac{\mathrm{d}F}{\mathrm{d}x}} % dF/dx
\newcommand{\dfdx}{\frac{\mathrm{d}f}{\mathrm{d}x}}  % df/dx
\newcommand{\interval}[1]{\left[ #1 \right]}

\newcommand{\norm}[1]{\left\| #1 \right\|}
\newcommand{\scalarprod}[1]{\left\langle #1 \right\rangle}
\newcommand{\vektor}[1]{\begin{pmatrix*}[r] #1 \end{pmatrix*}}
\newcommand{\dvektor}[1]{\begin{vmatrix*}[r] #1 \end{vmatrix*}}
\renewcommand{\span}[1]{\operatorname{span}\left(#1\right)}

\newcommand{\Nplus}{\mathbb{N}^+}
\newcommand{\N}{\mathbb{N}}
\newcommand{\Z}{\mathbb{Z}}
\newcommand{\Rnonneg}{\mathbb{R}^+_0}
\newcommand{\R}{\mathbb{R}}
\newcommand{\C}{\mathbb{C}}
\newcommand{\bigo}{\mathcal{O}}
\newcommand{\Pot}{\mathcal{P}}
\newcommand{\A}{\mathcal{A}}
\newcommand{\B}{\mathcal{B}}

\DeclareMathOperator{\img}{img}
\DeclareMathOperator{\defect}{defect}
\DeclareMathOperator{\rank}{rank}
\DeclareMathOperator{\trace}{trace}

\renewcommand{\solutiontitle}{\noindent\textbf{Lösung:}\par}


\makesavenoteenv{solution}
\lhead{Hausaufgabenblatt 08}
\rhead{Lineare Algebra 2}
\runningheadrule

\title{Lineare Algebra 2 \\ \large{Hausaufgabenblatt 08}}
\author{Patrick Gustav Blaneck}
\date{Abgabetermin: 23. Mai 2021}

\begin{document}
\maketitle
\begin{questions}
    \setcounter{question}{4}
    \question
    Berechnen Sie die Determinante der Matrix
    $$
        A = \vektor{4 & 5 & 6 \\ 2 & -2 & -1 \\ 0 & 1 & -3}
    $$
    \begin{parts}
        \part
        mit der Regel von Sarrus
        \begin{solution}
            $$
                \det A = 4 \cdot (-2) \cdot (-3) + 6 \cdot 2 \cdot 1 - 5 \cdot 2 \cdot (-3) - 4 \cdot 1 \cdot (-1) = 24 + 12 + 30 + 4 = 70
            $$\qed
        \end{solution}

        \part
        mit Hilfe des Gauß-Algorithmus
        \begin{solution}
            $$
                \vektor{4 & 5 & 6 \\ 2 & -2 & -1 \\ 0 & 1 & -3}
                \underset{}{\xrightarrow{\text{\Rnum{2}: \Rnum{2} + 2 \Rnum{3}}}}
                \vektor{4 & 5 & 6 \\ 2 & 0 & -7 \\ 0 & 1 & -3}
                \underset{}{\xrightarrow{\text{\Rnum{1}: \Rnum{1} - 5 \Rnum{3}}}}
                \vektor{4 & 0 & 21 \\ 2 & 0 & -7 \\ 0 & 1 & -3}
                \underset{}{\xrightarrow{\text{\Rnum{1}: \Rnum{1} - 2 \Rnum{2}}}}
                \vektor{0 & 0 & 35 \\ 2 & 0 & -7 \\ 0 & 1 & -3}
            $$
            $$
                \vektor{0 & 0 & 35 \\ 2 & 0 & -7 \\ 0 & 1 & -3}
                \underset{\frac{1}{35} \cdot \det(A)}{\xrightarrow{\text{\Rnum{1}: \Rnum{1} - 2 \Rnum{2}}}}
                \vektor{0 & 0 & 1 \\ 2 & 0 & -7 \\ 0 & 1 & -3}
                \underset{}{\xrightarrow{\text{(\Rnum{2}: \Rnum{2} + 7 \Rnum{1})} \ \land \ \text{(\Rnum{3}: \Rnum{3} + 3 \Rnum{1})}}}
                \vektor{0 & 0 & 1 \\ 2 & 0 & 0 \\ 0 & 1 & 0}
            $$
            $$
                \vektor{0 & 0 & 1 \\ 2 & 0 & 0 \\ 0 & 1 & 0}
                \underset{(-1)\cdot(-1)\cdot \det A}{\xrightarrow{\text{(\Rnum{1}$\,\leftrightarrow\,$\Rnum{3})} \ \land \ \text{(\Rnum{1}$\,\leftrightarrow\,$\Rnum{2})}}}
                \vektor{2 & 0 & 0 \\ 0 & 1 & 0 \\ 0 & 0 & 1}
            $$
            Es gilt:
            $$
                \det A = (-1) \cdot (-1) \cdot 35 \cdot \dvektor{2 & 0 & 0 \\ 0 & 1 & 0 \\ 0 & 0 & 1} = 35 \cdot 2 = 70
            $$\qed
        \end{solution}

        \newpage
        \part
        mit dem Entwicklungssatz
        \begin{solution}
            $$
                \det A = \dvektor{4 & 5 & 6 \\ 2 & -2 & -1 \\ \color{blue} 0 & \color{blue} 1 & \color{blue} -3} = - \textcolor{blue}{1} \cdot \underbrace{\dvektor{4 & 6 \\ 2 & -1}}_{-16} + \textcolor{blue}{(-3)} \cdot \underbrace{\dvektor{4 & 5 \\ 2 & -2}}_{-18} = 16 + 54 = 70
            $$\qed
        \end{solution}
    \end{parts}

    \newpage
    \question
    Eine spezielle $n\times n$-Tridiagonalmatrix $T_n$ ist gegeben durch:
    $$
        t_{i,j} = \begin{cases}
            2 & i = j                         \\
            1 & i = j-1 \text{ oder } j = i-1 \\
            0 & \text{sonst}
        \end{cases}
    $$
    \begin{parts}
        \part
        Schreiben Sie die Matrix für $n = 5$ explizit auf.
        \begin{solution}
            $$
                T_5 = \vektor{2 & 1 & 0 & 0 & 0 \\ 1 & 2 & 1 & 0 & 0 \\ 0 & 1 & 2 & 1 & 0 \\ 0 & 0 & 1 & 2 & 1 \\ 0 & 0 & 0 & 1 & 2}
            $$\qed
        \end{solution}

        \part
        Zeigen Sie, dass für die Determinanten $D_n = \det(T_n)$ gilt:
        $$
            D_n = 2D_{n-1} - D_{n-2}, \quad n = 2, 3, \ldots \ \text{mit} \ D_0 = 1.
        $$
        \begin{solution}
            Für die Berechnung von $D_n$ entwickeln wir einfach nach dem Laplace'schen Entwicklungssatz nach der ersten Zeile.
            Danach gilt bereits:
            $$
                D_n = t_{1, 1} \cdot \det(T_{n-1}) - t_{1,2} \cdot \det(T'_{n-1}) = 2\cdot \det(T_{n-1}) - 1 \cdot \det(T'_{n-1}) = 2D_{n-1} - 1\cdot \det(T'_{n-1})
            $$
            Sei hier $T'_{n}$ definiert mit
            $$
                t'_{i,j} = \begin{cases}
                    1       & i = j = 1                   \\
                    0       & i \neq j \text{ und } j = 1 \\
                    t_{i,j} & \text{sonst}
                \end{cases}
            $$

            Für $\det(T'_{n-1})$ entwickeln wir dann nach der ersten Spalte. Damit gilt:
            $$
                \det(T'_{n-1}) = t'_{1,1} \cdot \det(T_{n-2}) = 1\cdot \det(T_{n-2}) = D_{n-2}
            $$

            Insgesamt erhalten wir also:
            $$
                D_n = 2D_{n-1} - 1\cdot \det(T'_{n-1}) = 2D_{n-1} - D_{n-2}
            $$\qed
        \end{solution}

        \newpage
        \part
        Berechnen Sie $D_n$ mit Teil (b) für $n = 2, 3, 4, 5$.
        Geben Sie eine explizite (nicht rekursive) Formel für $D_n$ an und beweisen Sie sie.
        \begin{solution}
            $$
                \begin{aligned}
                    n = 2: \quad & D_2 = 2D_1 - D_0 = 2\cdot 2 - 1 \cdot 1 = 3 \\
                    n = 3: \quad & D_3 = 2D_2 - D_1 = 2\cdot 3 - 1 \cdot 2 = 4 \\
                    n = 4: \quad & D_4 = 2D_3 - D_2 = 2\cdot 4 - 1 \cdot 3 = 5 \\
                    n = 5: \quad & D_5 = 2D_4 - D_3 = 2\cdot 5 - 1 \cdot 4 = 6
                \end{aligned}
            $$

            Es gilt: $D_n = n+1$.

            Der Induktionsanfang ist bereits oben erledigt worden.

            Gelte nun $A(n) : D_n = n+1$ für ein festes, aber beliebiges $n \geq 2$, dann gilt für $A(n+1)$:
            $$
                D_{n+1} = n + 2 \iff 2 D_n - D_{n-1} = n+2 \overset{\text{IV}}{\iff} 2(n+1) - n = n + 2 \iff 2 = 2 \quad \checkmark
            $$
            Damit gilt $A(n)$ nach dem Prinzip der vollständigen Induktion für $n \geq 2$. \qed
        \end{solution}
    \end{parts}

    \newpage
    \question
    Es seien 3 Punkte mit den Koordinaten $(x_i, y_i)$ gegeben; es ist zu zeigen, dass
    $$
        \dvektor{
            x^2 + y^2 & x & y & 1 \\
            x_1^2 + y_1^2 & x_1 & y_1 & 1 \\
            x_2^2 + y_2^2 & x_2 & y_2 & 1 \\
            x_3^2 + y_3^2 & x_3 & y_3 & 1 \\
        }
        = 0
    $$
    die Gleichung des Kreises, der durch die 3 Punkte geht (Umkreis) ist, falls
    $$
        \dvektor{
            x_1 & y_1 & 1 \\
            x_2 & y_2 & 1 \\
            x_3 & y_3 & 1
        }
        \neq 0
    $$

    Was bedeutet die Bedingung geometrisch?
    \begin{solution}
        
    \end{solution}

    \newpage 
    \question 
    Berechnen Sie die Determinante der beiden folgenden Matrizen nach einer beliebigen Methode. 
    Es gibt in beiden Fällen eine sehr schnelle Methode. 
    \begin{parts}
        \part 
        $
        A = 
        \vektor{
            2 & -1 & 0 & 0 & -1 \\
            -1 & 2 & -1 & 0 & 0 \\
            0 & -1 & 2 & -1 & 0 \\
            0 & 0 & -1 & 2 & -1 \\
            -1 & 0 & 0 & -1 & 2
        }$
        \begin{solution}
            Wir erkennen schnell:
            $$
            \forall a_{i, 5}, \, i \in \interval{1, 5}_\N : a_{i, 5} = \sum^4_{j=1} a_{i, j}
            $$

            Damit ist die letzte Spalte der Matrix eine Linearkombination der anderen und damit gilt
            $$
            \det A = 0
            $$\qed
        \end{solution}

        \part 
        $
        B = 
        \vektor{
            1 & 1 & 1 \\
            a & b & c \\
            b+c & c+a & a+b
        }
        $
        \begin{solution}
            Mit der Umformung
            $$
            \vektor{
            1 & 1 & 1 \\
            a & b & c \\
            b+c & c+a & a+b
        }
        \underset{}{\xrightarrow{\text{\Rnum{3}: \Rnum{3} + \Rnum{2}}}}
        \vektor{
        1 & 1 & 1 \\
        a & b & c \\
        a+b+c & a+b+c & a+b+c
    }
            $$
            wird offensichtlich, dass die dritte Zeile lediglich ein Vielfaches von der ersten Zeile ist. 
            
            Damit sind die Zeilenvektoren linear abhängig und damit $\det B = 0$
        \end{solution}
    \end{parts}
\end{questions}
\end{document}