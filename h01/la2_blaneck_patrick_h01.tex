\documentclass[answers]{exam}

\usepackage[ngerman]{babel}
\usepackage{amsmath,amsthm,amsfonts,stmaryrd,amssymb,mathtools}
\usepackage{xcolor,soul}
\usepackage{polynom}
\usepackage{tikz}
\usepackage{footnote}
\usepackage{array}   % for \newcolumntype macro
\usepackage{pgfplots}
\usepgfplotslibrary{fillbetween}

\newcolumntype{L}{>{$}l<{$}} % math-mode version of "l" column type
\newcolumntype{R}{>{$}r<{$}} % math-mode version of "r" column type
\newcolumntype{C}{>{$}c<{$}} % math-mode version of "c" column type
\newcolumntype{P}{>{$}p<{$}} % math-mode version of "l" column type

\renewcommand*\env@matrix[1][*\c@MaxMatrixCols c]{%
  \hskip -\arraycolsep
  \let\@ifnextchar\new@ifnextchar
  \array{#1}}

\newcommand{\abs}[1]{\left| #1 \right|}
\newcommand{\cis}[1]{\left( \cos\left( #1 \right) + i \sin\left( #1 \right) \right)}
\newcommand{\sgn}{\text{sgn}} % Signum-Funktion
\newcommand{\diff}{\mathrm{d}} % Differentialquotienten d
\newcommand{\dx}{~\mathrm{d}x} % dx
\newcommand{\du}{~\mathrm{d}u} % du
\newcommand{\dv}{~\mathrm{d}v} % dv
\newcommand{\dw}{~\mathrm{d}w} % dw
\newcommand{\dt}{~\mathrm{d}t} % dt
\newcommand{\dn}{~\mathrm{d}n} % dn
\newcommand{\dudx}{~\frac{\mathrm{d}u}{\mathrm{d}x}} % du/dx
\newcommand{\dudn}{~\frac{\mathrm{d}u}{\mathrm{d}n}} % du/dn
\newcommand{\dvdx}{~\frac{\mathrm{d}v}{\mathrm{d}x}} % dv/dx
\newcommand{\dwdx}{~\frac{\mathrm{d}w}{\mathrm{d}x}} % dw/dx
\newcommand{\dtdx}{~\frac{\mathrm{d}t}{\mathrm{d}x}} % dt/dx
\newcommand{\ddx}{\frac{\mathrm{d}}{\mathrm{d}x}} % d/dx
\newcommand{\dFdx}{\frac{\mathrm{d}F}{\mathrm{d}x}} % dF/dx
\newcommand{\dfdx}{\frac{\mathrm{d}f}{\mathrm{d}x}}  % df/dx
\newcommand{\interval}[1]{\left[ #1 \right]}

\newcommand{\norm}[1]{\left\| #1 \right\|}
\newcommand{\scalarprod}[1]{\left\langle #1 \right\rangle}
\newcommand{\vektor}[1]{\begin{pmatrix*}[c] #1 \end{pmatrix*}}
\renewcommand{\span}[1]{\operatorname{span}\left(#1\right)}

\newcommand{\Nplus}{\mathbb{N}^+}
\newcommand{\N}{\mathbb{N}}
\newcommand{\Z}{\mathbb{Z}}
\newcommand{\Rnonneg}{\mathbb{R}^+_0}
\newcommand{\R}{\mathbb{R}}
\newcommand{\C}{\mathbb{C}}
\newcommand{\bigo}{\mathcal{O}}
\newcommand{\Pot}{\mathcal{P}}

\DeclareMathOperator{\im}{im}
\DeclareMathOperator{\defect}{def}
\DeclareMathOperator{\rg}{rg}

\renewcommand{\solutiontitle}{\noindent\textbf{Lösung:}\par}


\makesavenoteenv{solution}
\lhead{Hausaufgabenblatt 01}
\rhead{Lineare Algebra 2}
\runningheadrule

\title{Lineare Algebra 2 \\ \large{Hausaufgabenblatt 01}}
\author{Patrick Gustav Blaneck}
\date{Abgabetermin: 05. April 2021}

\begin{document}
\maketitle
\begin{questions}
    \setcounter{question}{4}
    \question
    Gegeben ist die Abbildung
    $$
        f : \R^2 \ni \vektor{x_1 \\ x_2} \to \vektor{x_1^2 \\ \frac{1}{2}x_2} \in \R^2
    $$
    sowie die Menge $A = [-1; 1] \times [0; 10]$.

    Bestimmen Sie $f\left(\vektor{1\\1}\right)$, $f^{-1} \left(\vektor{1\\1}\right)$, $f(A)$, $f^{-1}(A)$.
    \begin{solution}
        $$
            f\left(\vektor{1                  \\1}\right) = \vektor{1 \\ \frac{1}{2}}
        $$

        $$
            f^{-1} \left(\vektor{1            \\1}\right) = \left\{ \vektor{-1 \\ 2}, \vektor{1 \\ 2} \right\}
        $$

        $$
            f(A)       =  [0;1] \times [0;5]
        $$

        $$
            f^{-1}(A)  = [0;1] \times [0;20]
        $$
    \end{solution}

    \newpage
    \question
    \begin{parts}
        \part Sei $M \neq \emptyset$ und $\Pot(M)$ die Potenzmenge von $M$.
        Wir betrachten die Abbildung
        $$\varphi : M \to \Pot(M), \varphi(m) = \{m\} \text{ für } m \in M$$
        Beweisen oder widerlegen Sie, dass $\varphi$ injektiv bzw. surjektiv ist.
        \begin{solution}
            \emph{Injektivität:} $\varphi(m_1) = \varphi(m_2) \implies m_1 = m_2$
            $$
                \begin{aligned}
                                 & \varphi(m_1) = \varphi(m_2) &  & \implies m_1 = m_2                  \\
                    \equiv \quad & \{m_1\} = \{m_2\}           &  & \implies m_1 = m_2 \quad \checkmark
                \end{aligned}
            $$

            \emph{Surjektivität:} $\forall n \in \Pot(M), \exists m \in M : \varphi(m) = n \iff \forall n \in \Pot(M) : \varphi^{-1}(n) \neq \emptyset$
            $$
                M := \{0, 1\} \implies \varphi^{-1}(\{0, 1\}) = \emptyset \quad \lightning
            $$
        \end{solution}

        \part
        Zwei Mengen $M_1$, $M_2$ sind "`gleichmächtig"' im Sinne von Cantor ($\abs{M_1} = \abs{M_2}$), wenn es eine Bijektion zwischen $M_1$ und $M_2$ gibt.
        Sind die beiden Mengen endlich, impliziert dies, dass $M_1$ und $M_2$ gleich viele Elemente enthalten.
        Man zeige, dass es im Cantorschen Sinne "`so viele gerade wie natürliche Zahlen gibt"', indem man beweist, dass $\varphi : \N \to 2\N, \varphi(n)=2n$ eine Bijektion ist.
        \begin{solution}
            $\varphi : \N \to 2\N, \varphi(n)=2n$ ist genau dann eine Bijektion, wenn $\varphi$ injektiv und surjektiv ist.

            \emph{Injektivität:} $\varphi(x_1) = \varphi(x_2) \implies x_1 = x_2$
            $$
                \begin{aligned}
                                 & \varphi(x_1) = \varphi(x_2) &  & \implies x_1 = x_2                  \\
                    \equiv \quad & 2x_1 = 2x_2                 &  & \implies x_1 = x_2                  \\
                    \equiv \quad & x_1 = x_2                   &  & \implies x_1 = x_2 \quad \checkmark
                \end{aligned}
            $$

            \emph{Surjektivität:} $\forall y \in 2\N, \exists x \in \N : \varphi(x) = y$
            $$
                y \in 2\N \iff y = 2x, x\in\N \iff y = \varphi(x) \quad \checkmark
            $$
        \end{solution}

        \part
        Wir definieren
        $$
            f(n) = \begin{cases}
                -\frac{n-1}{2} & , n \text{ ungerade} \\
                \frac{n}{2}    & , n \text{ gerade}
            \end{cases},
            f : \N \to \Z
        $$
        Beweisen Sie: $f$ ist eine Bijektion. Was folgt daraus für die Mächtigkeit von $\N$, $2\N$ und $\Z$?
        \begin{solution}

        \end{solution}
    \end{parts}

    \newpage
    \question
    Welche der folgenden Abbildungen von $\R^2 \to \R^3$ sind linear?
    Bestimmen Sie gegebenenfalls den Kern.
    \begin{parts}
        \part $f_1(x_1, x_2) = \vektor{-x_2 \\ -x_1 \\ 5x_1-7x_2}$
        \begin{solution}
            $f_1$ ist genau dann \emph{linear}, wenn $f_1$ \emph{homogen} und \emph{additiv} ist.

            \emph{Homogenität:} $\forall x \in \R^2, \lambda \in \R: f_1(\lambda  x) = \lambda  f_1(x)$
            $$
                \begin{aligned}
                                 & f_1(\lambda  x)                 &  & = \lambda  f_1(x)        \\
                    \equiv \quad & f_1(\lambda  x_1, \lambda  x_2) &  & = \lambda  f_1(x_1, x_2) \\
                    \equiv \quad & \vektor{-\lambda  x_2                                         \\ -\lambda  x_1 \\ 5 \lambda  x_1-7  \lambda  x_2} &&= \lambda  \vektor{-x_2 \\ -x_1 \\ 5x_1-7x_2} \\
                    \equiv \quad & \lambda  \vektor{-x_2                                         \\ -x_1 \\ 5x_1-7x_2} &&= \lambda  \vektor{-x_2 \\ -x_1 \\ 5x_1-7x_2} \quad \checkmark
                \end{aligned}
            $$

            \emph{Additivität:} $\forall x, y \in \R^2: f_1(x + y) = f_1(x) + f_1(y)$
            $$
                \begin{aligned}
                                 & f_1(x + y)                &  & = f_1(x) + f_1(y)               \\
                    \equiv \quad & f_1(x_1 + y_1, x_2 + y_2) &  & = f_1(x_1, x_2) + f_1(y_1, y_2) \\
                    \equiv \quad & \vektor{-(x_2 + y_2)                                           \\ -(x_1 + y_1) \\ 5(x_1 + y_1)-7(x_2 + y_2)} &&= \vektor{-x_2 \\ -x_1 \\ 5x_1-7x_2} + \vektor{-y_2 \\ -y_1 \\ 5y_1-7y_2}\\
                    \equiv \quad & \vektor{-x_2-y_2                                               \\ -x_1 - y_1 \\ 5x_1 + 5y_1-7x_2 - 7y_2} &&= \vektor{-x_2-y_2 \\ -x_1 - y_1 \\ 5x_1 + 5y_1-7x_2 - 7y_2} \quad \checkmark
                \end{aligned}
            $$

            Damit ist $f_1$ linear. \qed

            $$
                \ker(f_1) = f_1^{-1}(0_{\R^3}) = \left\{ \vektor{0 \\ 0} \right\} = \left\{ 0_{\R^2} \right\}
            $$\qed
        \end{solution}

        \part $f_2(x_1, x_2) = \vektor{x_1 + 1 \\ x_2 - 1 \\ 3}$
        \begin{solution}
            $$
                f_2(0_{\R^2}) = \vektor{1 \\ -1 \\ 3} \neq 0_{\R^3} \quad \lightning
            $$
        \end{solution}

        \newpage
        \part $f_3(x_1, x_2) = \vektor{x_1  x_2 \\ 0 \\ x_1 + 1}$
        \begin{solution}
            $$
                f_3(0_{\R^2}) = \vektor{0 \\ 0 \\ 1} \neq 0_{\R^3} \quad \lightning
            $$
        \end{solution}

        \part $f_4(x_1, x_2) = \vektor{0 \\ x_1 - x_2 \\ x_1 + x_2}$
        \begin{solution}
            $f_4$ ist genau dann \emph{linear}, wenn $f_4$ \emph{homogen} und \emph{additiv} ist.

            \emph{Homogenität:} $\forall x \in \R^2, \lambda \in \R: f_4(\lambda  x) = \lambda  f_4(x)$
            $$
                \begin{aligned}
                                 & f_4(\lambda  x)                 &  & = \lambda  f_4(x)        \\
                    \equiv \quad & f_4(\lambda  x_1, \lambda  x_2) &  & = \lambda  f_4(x_1, x_2) \\
                    \equiv \quad & \vektor{0                                                     \\ \lambda  x_1 - \lambda  x_2 \\ \lambda  x_1 + \lambda  x_2} &&= \lambda  \vektor{0 \\ x_1 - x_2 \\ x_1 + x_2} \\
                    \equiv \quad & \lambda  \vektor{0                                            \\ x_1 - x_2 \\ x_1 + x_2} &&= \lambda  \vektor{0 \\ x_1 - x_2 \\ x_1 + x_2} \quad \checkmark
                \end{aligned}
            $$
            \emph{Additivität:} $\forall x, y \in \R^2: f_4(x + y) = f_4(x) + f_4(y)$
            $$
                \begin{aligned}
                                 & f_4(x + y)                &  & = f_4(x) + f_1(y)               \\
                    \equiv \quad & f_4(x_1 + y_1, x_2 + y_2) &  & = f_4(x_1, x_2) + f_4(y_1, y_2) \\
                    \equiv \quad & \vektor{0                                                      \\ x_1 + y_1 - x_2 - y_2 \\ x_1 + y_1 + x_2 + y_2} &&= \vektor{0 \\ x_1 - x_2 \\ x_1 + x_2} + \vektor{0 \\ y_1 - y_2 \\ y_1 + y_2}\\
                    \equiv \quad & \vektor{0                                                      \\ x_1 + y_1 - x_2 - y_2 \\ x_1 + y_1 + x_2 + y_2} &&= \vektor{0 \\ x_1 + y_1 - x_2 - y_2 \\ x_1 + y_1 + x_2 + y_2} \quad \checkmark
                \end{aligned}
            $$

            Damit ist $f_4$ linear. \qed

            $$
                \ker(f_4) = f_4^{-1}(0_{\R^3}) =\footnote{$x_1 - x_2 = x_1 + x_2 = 0 \implies x_1 = x_2 = 0$} \left\{ \vektor{0 \\ 0} \right\} = \left\{ 0_{\R^2} \right\}
            $$\qed
        \end{solution}
    \end{parts}

    \newpage
    \question
    Gegeben sei die Abbildung $f : \R^3 \to \R^2$ mit $f(x_1, x_2, x_3) = (x_1-2x_3, 4x_2)$
    \begin{parts}
        \part
        Zeigen Sie: $f$ ist linear.
        \begin{solution}
            $f$ ist genau dann \emph{linear}, wenn $f$ \emph{homogen} und \emph{additiv} ist.

            \emph{Homogenität:} $\forall x \in \R^3, \lambda \in \R: f(\lambda  x) = \lambda  f(x)$
            $$
                \begin{aligned}
                                 & f(\lambda  x)                                  &  & = \lambda  f(x)                              \\
                    \equiv \quad & f(\lambda  x_1, \lambda  x_2, \lambda  x_3)    &  & = \lambda  f(x_1, x_2, x_3)                  \\
                    \equiv \quad & (\lambda  x_1-2  \lambda  x_3, 4 \lambda  x_2) &  & = \lambda  (x_1-2x_3, 4x_2)                  \\
                    \equiv \quad & \lambda  (x_1-2x_3, 4x_2)                      &  & = \lambda  (x_1-2x_3, 4x_2) \quad \checkmark
                \end{aligned}
            $$
            \emph{Additivität:} $\forall x, y \in \R^3: f(x + y) = f(x) + f_4(y)$
            $$
                \begin{aligned}
                                 & f(x + y)                                 &  & = f(x) + f(y)                                               \\
                    \equiv \quad & f(x_1 + y_1, x_2 + y_2, x_3 + y_3)       &  & = f(x_1, x_2, x_3) + f(y_1, y_2, y_3)                       \\
                    \equiv \quad & (x_1 + y_1-2(x_3 + y_3), 4(x_2 + y_2))   &  & = (x_1-2x_3, 4x_2) + (y_1-2y_3, 4y_2)                       \\
                    \equiv \quad & (x_1 + y_1-2 x_3 -2  y_3, 4 x_2 +4  y_2) &  & = (x_1 + y_1-2 x_3 -2  y_3, 4 x_2 +4  y_2) \quad \checkmark
                \end{aligned}
            $$

            Damit ist $f$ linear. \qed
        \end{solution}

        \part
        Bestimmen Sie den Kern von $f$ und geben Sie $\dim(\ker(f))$ an.
        \begin{solution}
            $$
                f(x_1, x_2, x_3) = 0 \iff
                \begin{pmatrix}[ccc|c]
                    1 & 0 & -2 & 0 \\
                    0 & 4 & 0  & 0
                \end{pmatrix}
                \implies x_2 = 0 \land x_1 = 2x_3
            $$
            Daraus folgt:
            $$
                \ker(f) = f^{-1}(0) = \left\{ (2\lambda, 0, \lambda) \mid \lambda \in \R \right\} \implies \defect(f) = \dim(\ker(f)) = 1
            $$\qed
        \end{solution}

        \part
        Berechnen Sie die $\dim(\im(f))$ bzw. $\rg(f)$ und bestimmen Sie $\im(f)$.
        \begin{solution}
            Nach dem Rangsatz gilt:
            $$
                \dim \R^3 = \defect(f) + \rg(f) \implies \rg(f) = 2
            $$

            Sei $B = \{b_1, b_2, b_3\} = \left\{ (1, 0, 0), (0, 1, 0), (0, 0, 1) \right\}$ eine Basis des $\R^3$.
            Dann ist $f(B)$ ein Erzeugendensystem von $\im(f)$ (wenn $f$ isomorph ist, dann sogar eine Basis).
            $$
                \begin{aligned}
                    f(B) = \left\{ \vektor{1, 0}, \vektor{0, 4}, \vektor{-2, 0} \right\}
                \end{aligned}
            $$
            $f(b_1) = -2f(b_3) \implies \{f(b_1), f(b_2\} = \{ (1,0), (0, 4) \}$ ist eine Basis von $\im(f)$.\qed
        \end{solution}

        \newpage
        \part
        Ist die Abbildung $f$ injektiv oder surjektiv?
        \begin{solution}
            $$
                \ker(f)\neq \{0_{\R^3}\} \implies f \text{ ist nicht injektiv}
            $$

            $$
                \rg(f) = \dim(\R^2) \iff f \text{ ist surjektiv}
            $$\qed
        \end{solution}
    \end{parts}
\end{questions}
\end{document}