\documentclass[answers]{exam}

\usepackage[ngerman]{babel}
\usepackage{amsmath,amsthm,amsfonts,stmaryrd,amssymb,mathtools}
\usepackage{xcolor,soul}
\usepackage{polynom}
\usepackage{tikz}
\usepackage{footnote}
\usepackage{array}   % for \newcolumntype macro
\usepackage{pgfplots}
\usepgfplotslibrary{fillbetween}

\newcolumntype{L}{>{$}l<{$}} % math-mode version of "l" column type
\newcolumntype{R}{>{$}r<{$}} % math-mode version of "r" column type
\newcolumntype{C}{>{$}c<{$}} % math-mode version of "c" column type
\newcolumntype{P}{>{$}p<{$}} % math-mode version of "l" column type

\renewcommand*\env@matrix[1][*\c@MaxMatrixCols c]{%
  \hskip -\arraycolsep
  \let\@ifnextchar\new@ifnextchar
  \array{#1}}

  \newenvironment{sysmatrix}[1]
  {\left(\begin{array}{@{}#1@{}}}
  {\end{array}\right)}

\newcommand{\abs}[1]{\left| #1 \right|}
\newcommand{\cis}[1]{\left( \cos\left( #1 \right) + i \sin\left( #1 \right) \right)}
\newcommand{\sgn}{\text{sgn}} % Signum-Funktion
\newcommand{\diff}{\mathrm{d}} % Differentialquotienten d
\newcommand{\dx}{~\mathrm{d}x} % dx
\newcommand{\du}{~\mathrm{d}u} % du
\newcommand{\dv}{~\mathrm{d}v} % dv
\newcommand{\dw}{~\mathrm{d}w} % dw
\newcommand{\dt}{~\mathrm{d}t} % dt
\newcommand{\dn}{~\mathrm{d}n} % dn
\newcommand{\dudx}{~\frac{\mathrm{d}u}{\mathrm{d}x}} % du/dx
\newcommand{\dudn}{~\frac{\mathrm{d}u}{\mathrm{d}n}} % du/dn
\newcommand{\dvdx}{~\frac{\mathrm{d}v}{\mathrm{d}x}} % dv/dx
\newcommand{\dwdx}{~\frac{\mathrm{d}w}{\mathrm{d}x}} % dw/dx
\newcommand{\dtdx}{~\frac{\mathrm{d}t}{\mathrm{d}x}} % dt/dx
\newcommand{\ddx}{\frac{\mathrm{d}}{\mathrm{d}x}} % d/dx
\newcommand{\dFdx}{\frac{\mathrm{d}F}{\mathrm{d}x}} % dF/dx
\newcommand{\dfdx}{\frac{\mathrm{d}f}{\mathrm{d}x}}  % df/dx
\newcommand{\interval}[1]{\left[ #1 \right]}

\newcommand{\norm}[1]{\left\| #1 \right\|}
\newcommand{\scalarprod}[1]{\left\langle #1 \right\rangle}
\newcommand{\vektor}[1]{\begin{pmatrix*}[r] #1 \end{pmatrix*}}
\renewcommand{\span}[1]{\operatorname{span}\left(#1\right)}

\newcommand{\Nplus}{\mathbb{N}^+}
\newcommand{\N}{\mathbb{N}}
\newcommand{\Z}{\mathbb{Z}}
\newcommand{\Rnonneg}{\mathbb{R}^+_0}
\newcommand{\R}{\mathbb{R}}
\newcommand{\C}{\mathbb{C}}
\newcommand{\bigo}{\mathcal{O}}
\newcommand{\Pot}{\mathcal{P}}

\DeclareMathOperator{\im}{im}
\DeclareMathOperator{\defect}{def}
\DeclareMathOperator{\rg}{rg}

\renewcommand{\solutiontitle}{\noindent\textbf{Lösung:}\par}


\makesavenoteenv{solution}
\lhead{Hausaufgabenblatt 04}
\rhead{Lineare Algebra 2}
\runningheadrule

\title{Lineare Algebra 2 \\ \large{Hausaufgabenblatt 04}}
\author{Patrick Gustav Blaneck}
\date{Abgabetermin: 25. April 2021}

\begin{document}
\maketitle
\begin{questions}
    \setcounter{question}{4}
    \question
    Bestimmen Sie den Rang der zu den folgenden Matrizen gehörenden linearen Abbildungen:
    \begin{parts}
        \part  $A_1 = \vektor{-3 & 2 & 1 \\ -4 & 0 & -2}$
        \begin{solution}
            Wir berechnen zuerst den Kern der Matrix:
            $$
                \begin{sysmatrix}{ccc|c}
                    -3 & 2 & 1 & 0 \\
                    -4 & 0 & -2 & 0
                \end{sysmatrix}
                \sim
                \begin{sysmatrix}{ccc|c}
                    0 & 2 & \frac{5}{2} & 0 \\
                    -4 & 0 & -2 & 0
                \end{sysmatrix}
                \sim
                \begin{sysmatrix}{ccc|c}
                    0 & 4 & 5 & 0 \\
                    2 & 0 & 1 & 0
                \end{sysmatrix}
            $$

            Wir erhalten ein Gleichungssystem mit
            $$
                \begin{cases}
                    4x_2 + 5x_3 = 0 \\
                    2x_1 + x_3 = 0
                \end{cases}
            $$

            Sei $x_3 = \lambda$.
            Dann gilt:
            $$
                \begin{cases}
                    4x_2 + 5\lambda = 0 \\
                    2x_1 + \lambda = 0
                \end{cases}
                \implies
                \begin{cases}
                    x_2 = \frac{-5\lambda}{4} \\
                    x_1 = \frac{-\lambda}{2}
                \end{cases}
            $$

            Damit erhalten wir den Kern von $A_1$ mit
            $$
                \ker A_1 = \span{\vektor{-2 & -5 & 4}^T} \implies \defect A_2 = 1
            $$
            Damit gilt, dass $\rg A_1 = \dim \R^3 - \defect A_1 = 2$.\qed

            \vspace{2em}
            \textbf{Alternativ:}

            Wir sehen direkt, dass $\left\{\vektor{-3 & -4}^T, \vektor{2 & 0}^T\right\}$ linear unabhängig $\implies \rg A_1 = 2$.\qed
        \end{solution}

        \newpage

        \part $A_2 = \vektor{1 & 1 & 0 & 2 \\ 4 & 0 & 1 & 3 \\ 6 & 2 & 1 & 7 \\ 1 & 0 & 0 & 1}$
        \begin{solution}
            %Es gilt nach dem Laplace'schen Entwicklungssatz:
            %$$
            %    \begin{aligned}
            %        \det A_2 & = -1 \cdot \det \vektor{1               & 0 & 2 \\ 0 & 1 & 3 \\ 2 & 1 & 7} + 1 \cdot \det \vektor{1 & 1 & 0 \\ 4 & 0 & 1 \\ 6 & 2 & 1} \\
            %                 & = - \left( 1\cdot \vektor{1             & 3     \\ 1 & 7} + 2 \cdot \det \vektor{0 & 1 \\ 2 & 1} \right) + \left( 1\cdot \det \vektor{0 & 1 \\ 2 & 1} - 1\cdot \det \vektor{4 & 1 \\ 6 & 1} \right) \\
            %                 & = -(4 + 2 \cdot (-2)) + (-2 - (-2)) = 0
            %    \end{aligned}
            %$$

            Wir berechnen zuerst den Kern der Matrix:
            $$
                \begin{sysmatrix}{cccc|c}
                    1 & 1 & 0 & 2 & 0 \\
                    4 & 0 & 1 & 3 & 0 \\
                    6 & 2 & 1 & 7 & 0 \\
                    1 & 0 & 0 & 1 & 0
                \end{sysmatrix}
                \sim
                \begin{sysmatrix}{cccc|c}
                    0 & 1 & 0 & 1 & 0 \\
                    4 & 0 & 1 & 3 & 0 \\
                    6 & 2 & 1 & 7 & 0 \\
                    1 & 0 & 0 & 1 & 0
                \end{sysmatrix}
                \sim
                \begin{sysmatrix}{cccc|c}
                    0 & 1 & 0 & 1 & 0 \\
                    4 & 0 & 1 & 3 & 0 \\
                    6 & 0 & 1 & 5 & 0 \\
                    1 & 0 & 0 & 1 & 0
                \end{sysmatrix}
                \sim
                \begin{sysmatrix}{cccc|c}
                    0 & 1 & 0 & 1 & 0 \\
                    1 & 0 & 1 & 0 & 0 \\
                    0 & 0 & 0 & 0 & 0 \\
                    1 & 0 & 0 & 1 & 0
                \end{sysmatrix}^{(*)}
            $$

            Wir erhalten ein Gleichungssystem mit
            $$
                \begin{cases}
                    x_2 + x_4 = 0 \\
                    x_1 + x_3 = 0 \\
                    x_1 + x_4 = 0
                \end{cases}
            $$

            Sei $x_1 = \lambda$.
            Dann gilt:
            $$
                \begin{cases}
                    x_2 + x_4 = 0     \\
                    \lambda + x_3 = 0 \\
                    \lambda + x_4 = 0
                \end{cases}
                \implies
                \begin{cases}
                    x_2 + x_4 = 0  \\
                    x_3 = -\lambda \\
                    x_4 = -\lambda
                \end{cases}
                \implies
                \begin{cases}
                    x_2 = \lambda  \\
                    x_3 = -\lambda \\
                    x_4 = -\lambda
                \end{cases}
            $$

            Damit erhalten wir den Kern von $A_2$ mit
            $$
                \ker A_2 = \span{\vektor{1 & 1 & -1 & -1}^T} \implies \defect A_2 = 1
            $$

            Damit gilt, dass $\rg A_2 = \dim \R^4 - \defect A_2 = 3$.\qed

            \vspace{2em}
            \textbf{Alternativ:}

            Wir sehen direkt, dass in $(*)$ $\left\{\vektor{0 & 1 & 0 & 1}^T, \vektor{1 & 0 & 0 & 0}^T, \vektor{1 & 0 & 0 & 1}^T\right\}$ linear unabhängig $\implies \rg A_2 = 3$.\qed
        \end{solution}
    \end{parts}

    \newpage

    \question
    Bestimmen Sie den Rang der zu der folgenden Matrix gehörenden linearen Abbildung
    $$
        A = \vektor{1 & t & t^2 \\ t & 1 & t \\ t^2 & t & 1}
    $$
    in Abhängigkeit von $t$.
    \begin{solution}
        Wir bestimmen zuerst die Determinante von $A$:
        $$
            \det A = 1 + t^4 + t^4 - t^4 -t^2-t^2 = t^4 - 2t^2 + 1
        $$

        Wir wissen, dass $\rg A = 3 \iff \det A \neq 0$ (da $A \in \R^{3\times 3}$). Sei $u = t^2$, dann gilt:
        $$
            \det A = u^2 - 2u + 1 = 0 \implies u_{1,2} = 1 \pm \sqrt{1 - 1} = 1 \implies t \in \left\{-1, 1\right\} \quad \checkmark
        $$

        Damit wissen wir, dass $t \in \R \setminus \left\{-1, 1\right\} \implies \rg A = 3$.

        $$
            t = -1 \implies A = \vektor{1 & -1 & 1 \\ -1 & 1 & -1 \\ 1 & -1 & 1} \implies \rg A = 1
        $$
        $$
            t = 1 \implies A = \vektor{1 & 1 & 1 \\ 1 & 1 & 1 \\ 1 & 1 & 1} \implies \rg A = 1
        $$

        Damit wissen wir, dass $t \in \left\{-1, 1\right\} \implies \rg A = 1$.\qed
    \end{solution}

    \newpage

    \question
    Bestimmen Sie das Bild, den Rang, den Kern und die Dimension des Kerns der linearen Abbildung $f(x) = A\cdot x, f : \R^4 \to \R^4$ mit
    $$
        A = \vektor{1 & 3 & 0 & -2 \\ 2 & 1 & 4 & 1 \\ 0 & 1 & -1 & -1 \\ -1 & 0 & -2 & -1}.
    $$
    \begin{solution}
        Wir bestimmen zuerst den Rang der Matrix mithilfe eines Linearen Gleichungssystems:
        $$
            \begin{sysmatrix}{cccc|c}
                1 & 3 & 0 & -2 & 0 \\
                2 & 1 & 4 & 1 & 0 \\
                0 & 1 & -1 & -1 & 0 \\
                -1 & 0 & -2 & -1 & 0
            \end{sysmatrix}
            \sim
            \begin{sysmatrix}{cccc|c}
                0 & 3 & -2 & -3 & 0 \\
                2 & 1 & 4 & 1 & 0 \\
                0 & 1 & -1 & -1 & 0 \\
                -1 & 0 & -2 & -1 & 0
            \end{sysmatrix}
            \sim
            \begin{sysmatrix}{cccc|c}
                0 & 0 & 1 & 0 & 0 \\
                2 & 1 & 4 & 1 & 0 \\
                0 & 1 & -1 & -1 & 0 \\
                -1 & 0 & -2 & -1 & 0
            \end{sysmatrix}
        $$
        $$
            \sim
            \begin{sysmatrix}{cccc|c}
                0 & 0 & 1 & 0 & 0 \\
                2 & 1 & 0 & 1 & 0 \\
                0 & 1 & 0 & -1 & 0 \\
                -1 & 0 & 0 & -1 & 0
            \end{sysmatrix}
            \sim
            \begin{sysmatrix}{cccc|c}
                0 & 0 & 1 & 0 & 0 \\
                2 & 2 & 0 & 0 & 0 \\
                0 & 1 & 0 & -1 & 0 \\
                -1 & 0 & 0 & -1 & 0
            \end{sysmatrix}
            \sim
            \begin{sysmatrix}{cccc|c}
                0 & 0 & 1 & 0 & 0 \\
                1 & 1 & 0 & 0 & 0 \\
                0 & 1 & 0 & -1 & 0 \\
                1 & 0 & 0 & 1 & 0
            \end{sysmatrix}
        $$

        Wir erhalten ein Gleichungssystem mit
        $$
            \begin{cases}
                x_3 = 0       \\
                x_1 + x_2 = 0 \\
                x_2 - x_4 = 0 \\
                x_1 + x_4 = 0
            \end{cases}
        $$

        Sei $x_1 = \lambda$. Dann gilt:
        $$
            \begin{cases}
                x_3 = 0           \\
                \lambda + x_2 = 0 \\
                x_2 - x_4 = 0     \\
                \lambda + x_4 = 0
            \end{cases}
            \implies
            \begin{cases}
                x_3 = 0        \\
                x_2 = -\lambda \\
                x_2 - x_4 = 0  \\
                x_4 = -\lambda
            \end{cases}
            \implies
            \begin{cases}
                x_3 = 0        \\
                x_2 = -\lambda \\
                0 = 0          \\
                x_4 = -\lambda
            \end{cases}
        $$

        Damit erhalten wir den Kern von $A$ mit
        $$
            \ker A = \span{ \left\{\vektor{1 & -1 & 0 & -1}^T \right\}} \implies \defect A = 1.
        $$

        Damit gilt nach dem Rangsatz auch
        $$
            \rg A = \dim \R^4 - \defect A = 3,
        $$
        womit wir uns drei linear unabhängige Vektoren aus $A$ auswählen können, die dann eine Basis von $\im A$ ergeben.

        Wir wählen
        $$
            \im A = \span{\left\{ \vektor{1 & 2 & 0 & -1}^T, \vektor{3 & 1 & 1 & 0}^T, \vektor{0 & 4 & -1 & -2}^T \right\}}
        $$\qed
    \end{solution}

    \newpage

    \question
    Berechnen Sie in Abhängigkeit von $x$
    $$
        A = \vektor{2 & 1 & 2 & 3 \\ 2 & 5 & 4 & 3 \\ 1 & 2 & 5 & x \\ 2 & 1 & 3 & 5}
    $$
    \begin{parts}
        \part
        den Kern
        \begin{solution}
            Wir berechnen zuerst die Determinante von $A$:
            $$
                \begin{aligned}
                    \det A & = \det \vektor{1                   & 2 & 3 \\ 5 & 4 & 3 \\ 1 & 3 & 5} - 2\cdot\det \vektor{2 & 2 & 3 \\ 2 & 4 & 3 \\ 2 & 3 & 5} + 5\cdot\det \vektor{2 & 1 & 3 \\ 2 & 5 & 3 \\ 2 & 1 & 5} - x\cdot\det \vektor{2 & 1 & 2 \\ 2 & 5 & 4 \\ 2 & 1 & 3}\\
                           & = 0 - 2\cdot8 + 5\cdot16 - x\cdot8         \\                                                                     \\
                           & = 64-8x
                \end{aligned}
            $$

            Wir wissen, dass für $\det A = 64-8x \neq 0 \iff x \neq 8$ die Vektoren in $A$ linear unabhängig sind.
            Damit gilt $x\in \R\setminus\{8\} \implies \ker A = \{0\}$.

            Für $x=8$ gilt
            $$
                A = \vektor{2 & 1 & 2 & 3 \\ 2 & 5 & 4 & 3 \\ 1 & 2 & 5 & 8 \\ 2 & 1 & 3 & 5}.
            $$

            Wir berechnen nun den Kern von $A$ mithilfe eines Linearen Gleichungssystems:
            $$
                \begin{sysmatrix}{cccc|c}
                    2 & 1 & 2 & 3 & 0 \\
                    2 & 5 & 4 & 3 & 0 \\
                    1 & 2 & 5 & 8 & 0 \\
                    2 & 1 & 3 & 5 & 0
                \end{sysmatrix}
                \sim
                \begin{sysmatrix}{cccc|c}
                    2 & 1 & 2 & 3 & 0 \\
                    0 & 4 & 2 & 0 & 0 \\
                    1 & 2 & 5 & 8 & 0 \\
                    0 & 0 & 1 & 2 & 0
                \end{sysmatrix}
                \sim
                \begin{sysmatrix}{cccc|c}
                    2 & 1 & 2 & 3 & 0 \\
                    0 & 2 & 1 & 0 & 0 \\
                    1 & 0 & 4 & 8 & 0 \\
                    0 & 0 & 1 & 2 & 0
                \end{sysmatrix}
                \sim
                \begin{sysmatrix}{cccc|c}
                    0 & 1 & 0 & -1 & 0 \\
                    0 & 2 & 1 & 0 & 0 \\
                    1 & 0 & 0 & 0 & 0 \\
                    0 & 0 & 1 & 2 & 0
                \end{sysmatrix}
            $$

            Wir erhalten folgendes Gleichungssystem:
            $$
                \begin{cases}
                    x_2 - x_4 = 0  \\
                    2x_2 + x_3 = 0 \\
                    x_1 = 0        \\
                    x_3 + 2x_4 = 0
                \end{cases}
            $$

            Sei $x_2 = \lambda$. Dann gilt:
            $$
                \begin{cases}
                    \lambda - x_4 = 0  \\
                    2\lambda + x_3 = 0 \\
                    x_1 = 0            \\
                    x_3 + 2x_4 = 0
                \end{cases}
                \implies
                \begin{cases}
                    x_4 = \lambda   \\
                    x_3 = -2\lambda \\
                    x_1 = 0         \\
                    x_3 + 2x_4 = 0
                \end{cases}
                \implies
                \begin{cases}
                    x_4 = \lambda   \\
                    x_3 = -2\lambda \\
                    x_1 = 0         \\
                    0 = 0
                \end{cases}
            $$

            Damit erhalten wir den Kern von $A$ mit
            $$
                \ker A = \span{\{\vektor{0 & 1 & -2 & 1}\}^T}.
            $$\qed
        \end{solution}

        \part
        die Dimension des Kerns
        \begin{solution}
            $$
                x\in \R\setminus\{8\} \implies \ker A = \{0\} \implies \defect A = 0
            $$
            $$
                x = 8 \implies \ker A  = \span{\{\vektor{0 & 1 & -2 & 1}\}^T} \implies \defect A = 1
            $$\qed
        \end{solution}

        \part
        den Rang
        \begin{solution}
            $$
                x\in \R\setminus\{8\} \implies \defect A = 0 \implies \rg A = \dim \R^4 - \defect A = 4
            $$
            $$
                x = 8 \implies \defect A = 1 \implies \rg A = \dim \R^4 - \defect A = 3
            $$\qed
        \end{solution}

        \part
        das Bild
        \begin{solution}
            Wir wissen, dass wir uns immer $\abs{\rg A}$-viele linear unabhängige Vektoren aus $A$ aussuchen dürfen, um eine Basis von $\im A$ zu bilden.

            $$
                x\in \R\setminus\{8\} \implies\rg A = 4 \implies \span{\left\{ \vektor{2 \\ 2 \\ 1 \\ 2}, \vektor{1 \\ 5 \\ 2 \\ 1}, \vektor{2 \\ 4 \\ 5 \\ 3}, \vektor{3 \\ 3 \\ x \\ 5} \right\}} = \im A
            $$
            $$
                x = 8 \implies  \rg A = 3 \implies \span{\left\{ \vektor{2 \\ 2 \\ 1 \\ 2}, \vektor{1 \\ 5 \\ 2 \\ 1}, \vektor{2 \\ 4 \\ 5 \\ 3} \right\}} = \im A
            $$\qed
        \end{solution}
    \end{parts}
\end{questions}
\end{document}