\documentclass[answers]{exam}

\usepackage[ngerman]{babel}
\usepackage{amsmath,amsthm,amsfonts,stmaryrd,amssymb,mathtools}
\usepackage{xcolor,soul}
\usepackage{polynom}
\usepackage{tikz}
\usepackage{footnote}
\usepackage{nicefrac}
\usepackage{array}   % for \newcolumntype macro
\usepackage{pgfplots}
\usepgfplotslibrary{fillbetween}
\pgfplotsset{compat=1.8}

\newcolumntype{L}{>{$}l<{$}} % math-mode version of "l" column type
\newcolumntype{R}{>{$}r<{$}} % math-mode version of "r" column type
\newcolumntype{C}{>{$}c<{$}} % math-mode version of "c" column type
\newcolumntype{P}{>{$}p<{$}} % math-mode version of "l" column type

\renewcommand*\env@matrix[1][*\c@MaxMatrixCols c]{%
  \hskip -\arraycolsep
  \let\@ifnextchar\new@ifnextchar
  \array{#1}}

  \newenvironment{sysmatrix}[1]
  {\left(\begin{array}{@{}#1@{}}}
  {\end{array}\right)}

\newcommand{\abs}[1]{\left| #1 \right|}
\newcommand{\cis}[1]{\left( \cos\left( #1 \right) + i \sin\left( #1 \right) \right)}
\newcommand{\sgn}{\text{sgn}} % Signum-Funktion
\newcommand{\diff}{\mathrm{d}} % Differentialquotienten d
\newcommand{\dx}{~\mathrm{d}x} % dx
\newcommand{\du}{~\mathrm{d}u} % du
\newcommand{\dv}{~\mathrm{d}v} % dv
\newcommand{\dw}{~\mathrm{d}w} % dw
\newcommand{\dt}{~\mathrm{d}t} % dt
\newcommand{\dn}{~\mathrm{d}n} % dn
\newcommand{\dudx}{~\frac{\mathrm{d}u}{\mathrm{d}x}} % du/dx
\newcommand{\dudn}{~\frac{\mathrm{d}u}{\mathrm{d}n}} % du/dn
\newcommand{\dvdx}{~\frac{\mathrm{d}v}{\mathrm{d}x}} % dv/dx
\newcommand{\dwdx}{~\frac{\mathrm{d}w}{\mathrm{d}x}} % dw/dx
\newcommand{\dtdx}{~\frac{\mathrm{d}t}{\mathrm{d}x}} % dt/dx
\newcommand{\ddx}{\frac{\mathrm{d}}{\mathrm{d}x}} % d/dx
\newcommand{\dFdx}{\frac{\mathrm{d}F}{\mathrm{d}x}} % dF/dx
\newcommand{\dfdx}{\frac{\mathrm{d}f}{\mathrm{d}x}}  % df/dx
\newcommand{\interval}[1]{\left[ #1 \right]}

\newcommand{\norm}[1]{\left\| #1 \right\|}
\newcommand{\scalarprod}[1]{\left\langle #1 \right\rangle}
\newcommand{\vektor}[1]{\begin{pmatrix*}[r] #1 \end{pmatrix*}}
\renewcommand{\span}[1]{\operatorname{span}\left(#1\right)}

\newcommand{\Nplus}{\mathbb{N}^+}
\newcommand{\N}{\mathbb{N}}
\newcommand{\Z}{\mathbb{Z}}
\newcommand{\Rnonneg}{\mathbb{R}^+_0}
\newcommand{\R}{\mathbb{R}}
\newcommand{\C}{\mathbb{C}}
\newcommand{\bigo}{\mathcal{O}}
\newcommand{\Pot}{\mathcal{P}}
\newcommand{\A}{\mathcal{A}}
\newcommand{\B}{\mathcal{B}}

\DeclareMathOperator{\img}{img}
\DeclareMathOperator{\defect}{defect}
\DeclareMathOperator{\rank}{rank}
\DeclareMathOperator{\trace}{trace}

\renewcommand{\solutiontitle}{\noindent\textbf{Lösung:}\par}


\makesavenoteenv{solution}
\lhead{Hausaufgabenblatt 06}
\rhead{Lineare Algebra 2}
\runningheadrule

\title{Lineare Algebra 2 \\ \large{Hausaufgabenblatt 06}}
\author{Patrick Gustav Blaneck}
\date{Abgabetermin: 10. Mai 2021}

\begin{document}
\maketitle
\begin{questions}
    \setcounter{question}{4}
    \question
    $\A  = (e_1, e_2, e_3)$, $\A' = (a'_1, a'_2, a'_3)$ und $\A'' = (a''_1, a''_2, a''_3)$ bilden mit den kanonischen Einheitsvektoren $e_1, e_2, e_3$ sowie
    $$
        a'_1 = \vektor{1\\0\\0}, a'_2 = \vektor{1\\1\\0}, a'_3 = \vektor{1\\1\\1}
    $$
    bzw.
    $$
        a''_1 = \vektor{1\\-1\\0}, a''_2 = \vektor{-1\\0\\1}, a''_3 = \vektor{0\\1\\1}
    $$
    jeweils Basen des $\R^3$.
    \begin{parts}
        \part
        Bestimmen Sie die Transformationsmatrizen $T^{\A'}_{\A}$ sowie $T^{\A}_{\A'}$.
        \begin{solution}
            Wir wissen:
            $$
                \begin{aligned}
                    T^{\A'}_{\A} & = \A^{-1}\A'                                  \\
                    T^{\A}_{\A'} & = \A'^{-1}\A = \left(T^{\A'}_{\A}\right)^{-1}
                \end{aligned}
            $$

            Wir wissen auch, dass $\A = I \implies \A^{-1} = \A$.

            Berechnen von $\A'^{-1}$:
            $$
                \begin{sysmatrix}{rrr|rrr}
                    1 & 1 & 1 & 1 & 0 & 0 \\
                    0 & 1 & 1 & 0 & 1 & 0 \\
                    0 & 0 & 1 & 0 & 0 & 1
                \end{sysmatrix}
                \sim
                \begin{sysmatrix}{rrr|rrr}
                    1 & 1 & 0 & 1 & 0 & -1 \\
                    0 & 1 & 0 & 0 & 1 & -1 \\
                    0 & 0 & 1 & 0 & 0 & 1
                \end{sysmatrix}
                \sim
                \begin{sysmatrix}{rrr|rrr}
                    1 & 0 & 0 & 1 & -1 & 0 \\
                    0 & 1 & 0 & 0 & 1 & -1 \\
                    0 & 0 & 1 & 0 & 0 & 1
                \end{sysmatrix}
            $$

            Damit gilt:
            $$
                T^{\A'}_{\A} = \A^{-1}\A' = \vektor{1 & 0 & 0 \\ 0 & 1 & 0 \\ 0 & 0 & 1} \vektor{1 & 1 & 1 \\ 0 & 1 & 1 \\ 0 & 0 & 1} = \vektor{1 & 1 & 1 \\ 0 & 1 & 1 \\ 0 & 0 & 1}
            $$
            und
            $$
                T^{\A}_{\A'} = \A'^{-1}\A = \vektor{1 & -1 & 0 \\ 0 & 1 & -1 \\ 0 & 0 & 1} \vektor{1 & 0 & 0 \\ 0 & 1 & 0 \\ 0 & 0 & 1} = \vektor{1 & -1 & 0 \\ 0 & 1 & -1 \\ 0 & 0 & 1}.
            $$\qed
        \end{solution}

        \newpage
        \part
        Bestimmen Sie die Transformationsmatrizen $T^{\A''}_{\A}$ sowie $T^{\A}_{\A''}$.
        \begin{solution}
            Wir wissen:
            $$
                \begin{aligned}
                    T^{\A''}_{\A} & = \A^{-1}\A''                                   \\
                    T^{\A}_{\A''} & = \A''^{-1}\A = \left(T^{\A''}_{\A}\right)^{-1}
                \end{aligned}
            $$

            Berechnen von $\A''^{-1}$:
            $$
                \begin{sysmatrix}{rrr|rrr}
                    1 & -1 & 0 & 1 & 0 & 0 \\
                    -1 & 0 & 1 & 0 & 1 & 0 \\
                    0 & 1 & 1 & 0 & 0 & 1
                \end{sysmatrix}
                \sim
                \begin{sysmatrix}{rrr|rrr}
                    1 & 0 & 1 & 1 & 0 & 1 \\
                    -1 & 0 & 1 & 0 & 1 & 0 \\
                    0 & 1 & 1 & 0 & 0 & 1
                \end{sysmatrix}
                \sim
                \begin{sysmatrix}{rrr|rrr}
                    1 & 0 & 1 & 1 & 0 & 1 \\
                    0 & 0 & 1 & \nicefrac{1}{2} & \nicefrac{1}{2} & \nicefrac{1}{2} \\
                    0 & 1 & 1 & 0 & 0 & 1
                \end{sysmatrix}
                \sim
            $$
            $$
                \begin{sysmatrix}{rrr|rrr}
                    1 & 0 & 0 & \nicefrac{1}{2} & -\nicefrac{1}{2} & \nicefrac{1}{2} \\
                    0 & 1 & 0 & -\nicefrac{1}{2} & -\nicefrac{1}{2} & \nicefrac{1}{2}\\
                    0 & 0 & 1 & \nicefrac{1}{2} & \nicefrac{1}{2} & \nicefrac{1}{2}
                \end{sysmatrix}
                =
                \frac{1}{2}\vektor{1 & -1 & 1 \\ -1 & -1 & 1 \\ 1 & 1 & 1}
            $$
            Damit gilt:
            $$
                T^{\A''}_{\A} = \A^{-1}\A'' = \vektor{1 & 0 & 0 \\ 0 & 1 & 0 \\ 0 & 0 & 1} \vektor{1 & -1 & 0 \\ -1 & 0 & 1 \\ 0 & 1 & 1} = \vektor{1 & -1 & 0 \\ -1 & 0 & 1 \\ 0 & 1 & 1}
            $$
            und
            $$
                T^{\A}_{\A''} = \A''^{-1}\A = \frac{1}{2}\vektor{1 & -1 & 1 \\ -1 & -1 & 1 \\ 1 & 1 & 1} \vektor{1 & 0 & 0 \\ 0 & 1 & 0 \\ 0 & 0 & 1} = \frac{1}{2}\vektor{1 & -1 & 1 \\ -1 & -1 & 1 \\ 1 & 1 & 1}.
            $$\qed
        \end{solution}
        \part
        Bestimmen Sie die Transformationsmatrizen $T^{\A'}_{\A''}$ sowie $T^{\A''}_{\A'}$.
        \begin{solution}
            Wir wissen:
            $$
                \begin{aligned}
                    T^{\A'}_{\A''} & = \A''^{-1}\A'                                    \\
                    T^{\A''}_{\A'} & = \A'^{-1}\A'' = \left(T^{\A'}_{\A''}\right)^{-1}
                \end{aligned}
            $$

            Damit gilt:
            $$
                T^{\A'}_{\A''} = \A''^{-1}\A' = \frac{1}{2}\vektor{1 & -1 & 1 \\ -1 & -1 & 1 \\ 1 & 1 & 1} \vektor{1 & 1 & 1 \\ 0 & 1 & 1 \\ 0 & 0 & 1} = \frac{1}{2}\vektor{1 & 0 & 1 \\ -1 & -2 & -1 \\ 1 & 2 & 3}
            $$
            und
            $$
                T^{\A''}_{\A'} = \A'^{-1}\A'' = \vektor{1 & -1 & 0 \\ 0 & 1 & -1 \\ 0 & 0 & 1} \vektor{1 & -1 & 0 \\ -1 & 0 & 1 \\ 0 & 1 & 1} = \vektor{2 & -1 & -1 \\ -1 & -1 & 0 \\ 0 & 1 & 1}.
            $$\qed
        \end{solution}

        \newpage
        \part
        Bestimmen Sie die Koordinaten des Vektors $\vektor{1 & 0 & 1}^T$ bzgl. der Basen $\A'$ und $\A''$ unter Zuhilfenahme der in der Vorlesung benutzten Schreibweise.
        \begin{solution}
            Sei $x = \vektor{1 & 0 & 1}^T$.

            Dann gilt für $x$ bzgl. $\A'$:
            $$
                x = K_{\A'}(x) = T^{\A}_{\A'} \cdot K_{\A}(x) = \vektor{1 & -1 & 0 \\ 0 & 1 & -1 \\ 0 & 0 & 1} \vektor{1 \\ 0 \\ 1} = \vektor{1 \\ -1 \\ 1},
            $$
            bzw. bzgl. $\A''$:
            $$
                x = K_{\A''}(x) = T^{\A}_{\A''} \cdot K_{\A}(x) = \frac{1}{2}\vektor{1 & -1 & 1 \\ -1 & -1 & 1 \\ 1 & 1 & 1} \vektor{1 \\ 0 \\ 1} = \vektor{1 \\ 0 \\ 1}.
            $$\qed
        \end{solution}

    \end{parts}

    \newpage
    \question
    Die Kavalierprojektion dient dazu, dreidimensionale Objekte zweidimensional darzustellen.
    Die zugehörige Projektionsmatrix bzgl. der kanonischen Basen $\A$ und $\B$ lautet
    $$
        M^\A_\B = \vektor{1 & \nicefrac{\sqrt{3}}{2} & 0 \\ 0 & \nicefrac{1}{2} & 1}.
    $$
    Die Vektoren $(a'_1, a'_2, a'_3) = \left(\vektor{1 \\ 0 \\ 0}, \vektor{1 \\ 1 \\ 0}, \vektor{1 \\ 1 \\ 1}\right)$ spannen einen Spat auf und bilden die Basis $\A'$.
    \begin{parts}
        \part Zeichnen Sie den Spat.
        \begin{solution}

            \begin{center}
                \begin{tikzpicture}[scale=1]
                    \begin{axis}[
                            %view={10}{30},
                            width=20cm,
                            unit vector ratio*=1 1 1,
                            axis lines = middle,
                            ymin=-0.5,
                            ymax=3,
                            xmin=-0.5,
                            xmax=3,
                            zmin=-0.5,
                            zmax=2,
                            xlabel = $x$,
                            ylabel = $y$,
                            zlabel = $z$,
                            xtick distance=1,
                            ytick distance=1,
                            ztick distance=1,
                            xticklabels=\empty,
                            yticklabels=\empty,
                            zticklabels=\empty,
                            disabledatascaling,
                        ]

                        \node[above left] at (axis cs:0,0,0) {$A$};
                        \node[below] at (axis cs:1,0,0) {$B$};
                        \node[below] at (axis cs:2,1,0) {$C$};
                        \node[above left] at (axis cs:1,1,0) {$D$};
                        \node[above left] at (axis cs:1,1,1) {$E$};
                        \node[below right] at (axis cs:2,1,1) {$F$};
                        \node[above right] at (axis cs:3,2,1) {$G$};
                        \node[above left] at (axis cs:2,2,1) {$H$};

                        \draw[thick,red,->] (axis cs:0,0,0) -- (axis cs:1,0,0) node [midway, below] {$a'_1$};
                        \draw[thick,blue,->,dashed] (axis cs:0,0,0) -- (axis cs:1,1,0) node [midway, above left] {$a'_2$};
                        \draw[thick,purple,->] (axis cs:0,0,0) -- (axis cs:1,1,1) node [midway, below right] {$a'_3$};



                        \draw (axis cs:1,0,0) -- (axis cs:2,1,0);
                        \draw[dashed] (axis cs:2,1,0) -- (axis cs:1,1,0);
                        \draw (axis cs:1,1,1) -- (axis cs:2,1,1) -- (axis cs:3,2,1) -- (axis cs: 2,2,1) -- (axis cs:1,1,1);
                        \draw (axis cs:1,0,0) -- (axis cs:2,1,1);
                        \draw (axis cs:2,1,0) -- (axis cs:3,2,1);
                        \draw[dashed] (axis cs:1,1,0) -- (axis cs:2,2,1);

                    \end{axis}

                \end{tikzpicture}
            \end{center}

        \end{solution}

        \newpage
        \part Bestimmen Sie alle acht Eckpunkte des Spates sowie dessen Volumen.
        \begin{solution}
            Seien die Eckpunkte $A, B, \ldots, H$ und ihre Ortsvektoren $\vec{a}, \vec{b}, \ldots, \vec{h}$ gegeben mit:
            $$
                \begin{aligned}
                    A: \vec{a} & = \vektor{0                      \\0\\0} \\
                    B: \vec{b} & = a'_1 = \vektor{1               \\0\\0} \\
                    C: \vec{c} & = a'_1 + a'_2 = \vektor{1        \\0\\0} + \vektor{1\\1\\0} = \vektor{2\\1\\0}\\
                    D: \vec{d} & = a'_2 = \vektor{1               \\1\\0} \\
                    E: \vec{e} & = a'_3 = \vektor{1               \\1\\1} \\
                    F: \vec{f} & = a'_3 + a'_1 = \vektor{1        \\1\\0} + \vektor{1\\0\\0} = \vektor{2\\1\\1} \\
                    G: \vec{g} & = a'_3 + a'_1 + a'_2 = \vektor{1 \\1\\1} + \vektor{1 \\0\\0} + \vektor{1\\1\\0} = \vektor{3\\2\\1} \\
                    H: \vec{h} & = a'_3 + a'_2= \vektor{1         \\1\\1} + \vektor{1\\1\\0} = \vektor{2\\2\\1} \\
                \end{aligned}
            $$

            Das Volumen des Spats ist bekanntermaßen gegeben mit:
            $$
                V = \scalarprod{a'_1 \times a'_2, a'_3}  = \scalarprod{\vektor{1\\0\\0} \times \vektor{1\\1\\0}, \vektor{1\\1\\1}} = \scalarprod{\vektor{0\\0\\1}, \vektor{1\\1\\1}} = 1
            $$\qed
        \end{solution}

        \newpage
        \part Projizieren Sie alle Eckpunkte des Spates mit Hilfe der Kavalierprojektion in die $\R^2$-Ebene.
        \begin{solution}
            Sei $x \in \R^3$ ein beliebiger Ortsvektor eines Eckpunktes des Spats.

            Dann ist offensichtlich
            $$
                M^\A_\B \cdot K_{\A}(x)
            $$
            der projizierte Vektor im $\R^2$.

            Angewandt auf alle Ortsvektoren bedeutet das:
            $$
                \begin{aligned}
                    A': \vec{a'} & = M^\A_\B \cdot K_{\A}(\vec{a}) = \vektor{1 & \nicefrac{\sqrt{3}}{2} & 0 \\ 0 & \nicefrac{1}{2} & 1} \cdot \vektor{0 \\ 0 \\ 0} = \vektor{0 \\ 0} \\
                    B': \vec{b'} & = M^\A_\B \cdot K_{\A}(\vec{b}) = \vektor{1 & \nicefrac{\sqrt{3}}{2} & 0 \\ 0 & \nicefrac{1}{2} & 1} \cdot \vektor{1 \\ 0 \\ 0} = \vektor{1 \\ 0} \\
                    C': \vec{c'} & = M^\A_\B \cdot K_{\A}(\vec{c}) = \vektor{1 & \nicefrac{\sqrt{3}}{2} & 0 \\ 0 & \nicefrac{1}{2} & 1} \cdot \vektor{2 \\ 1 \\ 0} = \vektor{2+\nicefrac{\sqrt{3}}{2} \\ \nicefrac{1}{2}} \\
                    D': \vec{d'} & = M^\A_\B \cdot K_{\A}(\vec{d}) = \vektor{1 & \nicefrac{\sqrt{3}}{2} & 0 \\ 0 & \nicefrac{1}{2} & 1} \cdot \vektor{1 \\ 1 \\ 0} = \vektor{1+\nicefrac{\sqrt{3}}{2} \\ \nicefrac{1}{2}} \\
                    E': \vec{e'} & = M^\A_\B \cdot K_{\A}(\vec{e}) = \vektor{1 & \nicefrac{\sqrt{3}}{2} & 0 \\ 0 & \nicefrac{1}{2} & 1} \cdot \vektor{1 \\ 1 \\ 1} = \vektor{1+\nicefrac{\sqrt{3}}{2} \\ \nicefrac{3}{2}} \\
                    F': \vec{f'} & = M^\A_\B \cdot K_{\A}(\vec{f}) = \vektor{1 & \nicefrac{\sqrt{3}}{2} & 0 \\ 0 & \nicefrac{1}{2} & 1} \cdot \vektor{2 \\ 1 \\ 1} = \vektor{2+\nicefrac{\sqrt{3}}{2} \\ \nicefrac{3}{2}} \\
                    G': \vec{g'} & = M^\A_\B \cdot K_{\A}(\vec{g}) = \vektor{1 & \nicefrac{\sqrt{3}}{2} & 0 \\ 0 & \nicefrac{1}{2} & 1} \cdot \vektor{3 \\ 2 \\ 1} = \vektor{3+\sqrt{3} \\ 2} \\
                    H': \vec{h'} & = M^\A_\B \cdot K_{\A}(\vec{h}) = \vektor{1 & \nicefrac{\sqrt{3}}{2} & 0 \\ 0 & \nicefrac{1}{2} & 1} \cdot \vektor{2 \\ 2 \\ 1} = \vektor{2+\sqrt{3} \\ 2}
                \end{aligned}
            $$
        \end{solution}

        \newpage
        \part Neben der kanonischen Basis $\B$ gibt es im $\R^2$ auch noch die Basis $\B' = (b'_1, b'_2)$ mit
        $$
            b'_1 = \vektor{1\\0}, b'_2 = \vektor{1\\1}.
        $$
        Bestimmen Sie die Abbildungsmatrix $M^{\A'}_{\B'}$ bzgl. der Basen $\A'$ und $\B'$.
        \begin{solution}
            Es gilt:
            $$
                \begin{aligned}
                    M^{\A'}_{\B'} & =T^{\B}_{\B'} \cdot M^{\A}_{\B} \cdot T^{\A'}_{\A}                                                         \\
                                  & = \vektor{1                                        & 1                                                     \\ 0 & 1}^{-1} \cdot \vektor{1 & \nicefrac{\sqrt{3}}{2} & 0 \\ 0 & \nicefrac{1}{2} & 1} \cdot \vektor{1 & 1 & 1 \\ 0 & 1 & 1 \\ 0 & 0 & 1} \\
                                  & = \vektor{1                                        & -1                                                    \\ 0 & 1} \cdot \vektor{1 & \nicefrac{\sqrt{3}}{2} & 0 \\ 0 & \nicefrac{1}{2} & 1} \cdot \vektor{1 & 1 & 1 \\ 0 & 1 & 1 \\ 0 & 0 & 1} \\
                                  & = \vektor{1                                        & \nicefrac{\sqrt{3} - 1}{2} & -1                       \\ 0 & \nicefrac{1}{2} & 1} \cdot \vektor{1 & 1 & 1 \\ 0 & 1 & 1 \\ 0 & 0 & 1} \\
                                  & = \vektor{1                                        & \nicefrac{\sqrt{3} + 1}{2} & \nicefrac{\sqrt{3}-1}{2} \\ 0 & \nicefrac{1}{2} & \nicefrac{3}{2}}
                \end{aligned}
            $$\qed
        \end{solution}
    \end{parts}

    \newpage
    \question
    Gegeben sei die Basis $\B = (1, x, x^2, x^3)$ der Polynome vom Grad $\leq 3$.
    Stellen Sie die Transformationsmatrix des Basiswechsels von $\B$ nach $\mathcal{C}$ auf mit
    $$
        \mathcal{C} = (1, x-c, (x-c)^2, (x-c)^3), \qquad c\in\R .
    $$
    \begin{solution}
        Es gilt:
        $$
            \B = \vektor{1 & 0 & 0 & 0 \\ 0 & 1 & 0 & 0 \\ 0 & 0 & 1 & 0 \\ 0 & 0 & 0 & 1} \quad \text{und} \quad \mathcal{C} = \vektor{1 & -c & c^2 & -c^3 \\ 0 & 1 & -2c & 3c^2 \\ 0 & 0 & 1 & -3c \\ 0 & 0 & 0 & 1}.
        $$
        Auch wissen wir:
        $$
            T^\B_\mathcal{C} = \mathcal{C}^{-1} \cdot \B.
        $$
        Fehlt uns also noch die Inverse von $\mathcal{C}$:
        $$
            \begin{sysmatrix}{rrrr|rrrr}
                1 & -c & c^2 & -c^3 & 1 & 0 & 0 & 0 \\
                0 & 1 & -2c & 3c^2 & 0 & 1 & 0 & 0  \\
                0 & 0 & 1 & -3c & 0 & 0 & 1 & 0  \\
                0 & 0 & 0 & 1 & 0 & 0 & 0 & 1
            \end{sysmatrix}
            \sim
            \begin{sysmatrix}{rrrr|rrrr}
                1 & -c & c^2 & -c^3 & 1 & 0 & 0 & 0 \\
                0 & 1 & -2c & 3c^2 & 0 & 1 & 0 & 0  \\
                0 & 0 & 1 & 0 & 0 & 0 & 1 & 3c  \\
                0 & 0 & 0 & 1 & 0 & 0 & 0 & 1
            \end{sysmatrix}
        $$
        $$
            \sim
            \begin{sysmatrix}{rrrr|rrrr}
                1 & -c & c^2 & -c^3 & 1 & 0 & 0 & 0 \\
                0 & 1 & -2c & 0 & 0 & 1 & 0 & -3c^2  \\
                0 & 0 & 1 & 0 & 0 & 0 & 1 & 3c  \\
                0 & 0 & 0 & 1 & 0 & 0 & 0 & 1
            \end{sysmatrix}
            \sim
            \begin{sysmatrix}{rrrr|rrrr}
                1 & -c & c^2 & -c^3 & 1 & 0 & 0 & 0 \\
                0 & 1 & 0 & 0 & 0 & 1 & 2c & 3c^2  \\
                0 & 0 & 1 & 0 & 0 & 0 & 1 & 3c  \\
                0 & 0 & 0 & 1 & 0 & 0 & 0 & 1
            \end{sysmatrix}
        $$
        $$
            \sim
            \begin{sysmatrix}{rrrr|rrrr}
                1 & -c & c^2 & 0 & 1 & 0 & 0 & c^3 \\
                0 & 1 & 0 & 0 & 0 & 1 & 2c & 3c^2  \\
                0 & 0 & 1 & 0 & 0 & 0 & 1 & 3c  \\
                0 & 0 & 0 & 1 & 0 & 0 & 0 & 1
            \end{sysmatrix}
            \sim
            \begin{sysmatrix}{rrrr|rrrr}
                1 & -c & 0 & 0 & 1 & 0 & -c^2 & -2c^3 \\
                0 & 1 & 0 & 0 & 0 & 1 & 2c & 3c^2  \\
                0 & 0 & 1 & 0 & 0 & 0 & 1 & 3c  \\
                0 & 0 & 0 & 1 & 0 & 0 & 0 & 1
            \end{sysmatrix}
        $$
        $$
            \sim
            \begin{sysmatrix}{rrrr|rrrr}
                1 & 0 & 0 & 0 & 1 & c & c^2 & c^3 \\
                0 & 1 & 0 & 0 & 0 & 1 & 2c & 3c^2  \\
                0 & 0 & 1 & 0 & 0 & 0 & 1 & 3c  \\
                0 & 0 & 0 & 1 & 0 & 0 & 0 & 1
            \end{sysmatrix}
        $$

        Damit gilt schließlich:
        $$
            T^\B_\mathcal{C} = \mathcal{C}^{-1} \cdot \B = \vektor{1 & c & c^2 & c^3 \\ 0 & 1 & 2c & 3c^2 \\ 0 & 0 & 1 & 3c \\ 0 & 0 & 0 & 1} \cdot \vektor{1 & 0 & 0 & 0 \\ 0 & 1 & 0 & 0 \\ 0 & 0 & 1 & 0 \\ 0 & 0 & 0 & 1} = \vektor{1 & c & c^2 & c^3 \\ 0 & 1 & 2c & 3c^2 \\ 0 & 0 & 1 & 3c \\ 0 & 0 & 0 & 1}
        $$\qed
    \end{solution}

    \newpage
    \question
    Die untenstehende Transformationsmatrix für den $\R^2$ wurde durch eine Drehung, eine Spiegelung an einer Achse und eine Verzerrung (in dieser Reihenfolge) erstellt.
    Geben Sie durch die drei einzelnen Transformationsmatrizen eine mögliche Lösung für die Zerlegung der Matrix an.
    $$
        T = \vektor{-\nicefrac{3\sqrt{3}}{2} & \nicefrac{3}{2} \\ \nicefrac{1}{2} & \nicefrac{\sqrt{3}}{2} }
    $$

    \emph{Hinweis:} Eine Verzerrung ist von der Form
    $$
        \vektor{a & 0 \\ 0 & b}, \quad a,b>0,
    $$
    d.h. in $x$-Richtung bzw. $y$-Richtung wird um den Faktor $a$ bzw. $b$ gedehnt oder gestaucht.
    \begin{solution}
        Wir wissen, dass eine Drehung/Rotation um den Winkel $\phi$ durch
        $$
            R_\phi = \vektor{\cos\phi & -\sin\phi \\ \sin\phi & \cos\phi},
        $$
        eine Spiegelung an der $y$-Achse durch
        $$
            S_y = \vektor{-1 & 0 \\ 0 & 1}
        $$
        und eine Verzerrung um $a$ in $x$-Richtung bzw. $b$ in $y$-Richtung dargestellt werden kann durch
        $$
            V_v = \vektor{a & 0 \\ 0 & b} \text{ mit } v = \vektor{a \\ b}.
        $$

        Wir wissen aus der Aufgabe, dass gilt:
        $$
            T = V_v \cdot S_y \cdot R_\phi = \vektor{a & 0 \\ 0 & b} \cdot \vektor{-1 & 0 \\ 0 & 1} \cdot \vektor{\cos\phi & -\sin\phi \\ \sin\phi & \cos\phi}
        $$

        Wir berechnen:
        $$
            \begin{aligned}
                             & T                                &                 & = V_v \cdot S_y \cdot R \\
                \equiv \quad & \vektor{-\nicefrac{3\sqrt{3}}{2} & \nicefrac{3}{2}                           \\ \nicefrac{1}{2} & \nicefrac{\sqrt{3}}{2} } && = \vektor{a & 0 \\ 0 & b} \cdot \vektor{-1 & 0 \\ 0 & 1} \cdot \vektor{\cos\phi & -\sin\phi \\ \sin\phi & \cos\phi} \\
                \equiv \quad & \vektor{-\nicefrac{3\sqrt{3}}{2} & \nicefrac{3}{2}                           \\ \nicefrac{1}{2} & \nicefrac{\sqrt{3}}{2} } && = \vektor{-a & 0 \\ 0 & b} \cdot \vektor{\cos\phi & -\sin\phi \\ \sin\phi & \cos\phi} \\
                \equiv \quad & \vektor{-\nicefrac{3\sqrt{3}}{2} & \nicefrac{3}{2}                           \\ \nicefrac{1}{2} & \nicefrac{\sqrt{3}}{2} } && = \vektor{-a \cos\phi & a\sin\phi \\ b\sin\phi & b\cos\phi}
            \end{aligned}
        $$
        Wir erhalten die Gleichungen\footnote{Sei $\phi \notin \left\{ k \pi \mid k \in \frac{\Z}{2}\right\}$}:
        $$
            \begin{cases}
                -a\cos\phi & = -\frac{3\sqrt{3}}{2} \\
                a\sin\phi  & = \frac{3}{2}          \\
                b\sin\phi  & = \frac{1}{2}          \\
                b\cos\phi  & = \frac{\sqrt{3}}{2}
            \end{cases}
            \implies
            \begin{cases}
                a & = \frac{3\sqrt{3}}{2\cos\phi} = \frac{3}{2\sin\phi} \\
                b & = \frac{1}{2\sin\phi} = \frac{\sqrt{3}}{2\cos\phi}
            \end{cases}
        $$

        Wir erhalten weiterhin:\footnote{Analog für die andere Gleichung.}
        $$
            \begin{aligned}
                               & \frac{3\sqrt{3}}{2\cos\phi} &  & = \frac{3}{2\sin\phi}        \\
                \equiv \quad   & \frac{\sin\phi}{\cos\phi}   &  & = \frac{1}{\sqrt{3}}         \\
                \equiv \quad   & \tan\phi                    &  & = \frac{1}{\sqrt{3}}         \\
                \implies \quad & \phi                        &  & = \arctan \frac{1}{\sqrt{3}} \\
                \equiv \quad   & \phi                        &  & = \frac{\pi}{6}
            \end{aligned}
        $$

        Einsetzen für $a$ bzw. $b$ ergibt:
        $$
            \begin{aligned}
                a & = \frac{3\sqrt{3}}{2\cos\phi} \implies a = 3 \\
                b & = \frac{1}{2\sin\phi} \implies b = 1
            \end{aligned}
        $$

        Damit ergibt sich schließlich eine Zerlegung für $T$ mit $v = \vektor{3\\1}$ und $\phi = \frac{\pi}{6}$ mit:
        $$
            T = V_{v} \cdot S_y \cdot R_\phi = \vektor{3 & 0 \\ 0 & 1} \cdot \vektor{-1 & 0 \\ 0 & 1} \cdot \vektor{\frac{\sqrt{3}}{2} & -\frac{1}{2} \\ \frac{1}{2} & \frac{\sqrt{3}}{2}}.
        $$\qed
    \end{solution}
\end{questions}
\end{document}