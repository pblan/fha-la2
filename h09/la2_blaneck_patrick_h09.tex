\documentclass[answers]{exam}

\usepackage[ngerman, shorthands=off]{babel}
\usepackage{amsmath,amsthm,amsfonts,stmaryrd,amssymb,mathtools}
\usepackage{xcolor,soul}
\usepackage{polynom}
\usepackage{tikz}
\usetikzlibrary{arrows.meta,angles,quotes,calc}
\usepackage{footnote}
\usepackage{nicefrac}
\usepackage{siunitx}
\usepackage{array}   % for \newcolumntype macro
\usepackage{pgfplots}
\usepgfplotslibrary{fillbetween}
\pgfplotsset{compat=1.8}

\newcolumntype{L}{>{$}l<{$}} % math-mode version of "l" column type
\newcolumntype{R}{>{$}r<{$}} % math-mode version of "r" column type
\newcolumntype{C}{>{$}c<{$}} % math-mode version of "c" column type
\newcolumntype{P}{>{$}p<{$}} % math-mode version of "l" column type

\renewcommand*\env@matrix[1][*\c@MaxMatrixCols c]{%
  \hskip -\arraycolsep
  \let\@ifnextchar\new@ifnextchar
  \array{#1}}

  \newenvironment{sysmatrix}[1]
  {\left(\begin{array}{@{}#1@{}}}
  {\end{array}\right)}

  \newcommand{\Rnum}[1]{\uppercase\expandafter{\romannumeral #1\relax}}
\newcommand{\abs}[1]{\left| #1 \right|}
\newcommand{\cis}[1]{\left( \cos\left( #1 \right) + i \sin\left( #1 \right) \right)}
\newcommand{\sgn}{\text{sgn}} % Signum-Funktion
\newcommand{\diff}{\mathrm{d}} % Differentialquotienten d
\newcommand{\dx}{~\mathrm{d}x} % dx
\newcommand{\du}{~\mathrm{d}u} % du
\newcommand{\dv}{~\mathrm{d}v} % dv
\newcommand{\dw}{~\mathrm{d}w} % dw
\newcommand{\dt}{~\mathrm{d}t} % dt
\newcommand{\dn}{~\mathrm{d}n} % dn
\newcommand{\dudx}{~\frac{\mathrm{d}u}{\mathrm{d}x}} % du/dx
\newcommand{\dudn}{~\frac{\mathrm{d}u}{\mathrm{d}n}} % du/dn
\newcommand{\dvdx}{~\frac{\mathrm{d}v}{\mathrm{d}x}} % dv/dx
\newcommand{\dwdx}{~\frac{\mathrm{d}w}{\mathrm{d}x}} % dw/dx
\newcommand{\dtdx}{~\frac{\mathrm{d}t}{\mathrm{d}x}} % dt/dx
\newcommand{\ddx}{\frac{\mathrm{d}}{\mathrm{d}x}} % d/dx
\newcommand{\dFdx}{\frac{\mathrm{d}F}{\mathrm{d}x}} % dF/dx
\newcommand{\dfdx}{\frac{\mathrm{d}f}{\mathrm{d}x}}  % df/dx
\newcommand{\interval}[1]{\left[ #1 \right]}

\newcommand{\norm}[1]{\left\| #1 \right\|}
\newcommand{\scalarprod}[1]{\left\langle #1 \right\rangle}
\newcommand{\vektor}[1]{\begin{pmatrix*}[r] #1 \end{pmatrix*}}
\newcommand{\dvektor}[1]{\begin{vmatrix*}[r] #1 \end{vmatrix*}}
\renewcommand{\span}[1]{\operatorname{span}\left(#1\right)}

\newcommand{\Nplus}{\mathbb{N}^+}
\newcommand{\N}{\mathbb{N}}
\newcommand{\Z}{\mathbb{Z}}
\newcommand{\Rnonneg}{\mathbb{R}^+_0}
\newcommand{\R}{\mathbb{R}}
\newcommand{\C}{\mathbb{C}}
\newcommand{\bigo}{\mathcal{O}}
\newcommand{\Pot}{\mathcal{P}}
\newcommand{\A}{\mathcal{A}}
\newcommand{\B}{\mathcal{B}}

\DeclareMathOperator{\img}{img}
\DeclareMathOperator{\defect}{defect}
\DeclareMathOperator{\rank}{rank}
\DeclareMathOperator{\trace}{trace}
\DeclareMathOperator{\Sol}{Sol}
\DeclareMathOperator{\row}{row}
\DeclareMathOperator{\col}{col}

\renewcommand{\solutiontitle}{\noindent\textbf{Lösung:}\par}


\makesavenoteenv{solution}
\lhead{Hausaufgabenblatt 09}
\rhead{Lineare Algebra 2}
\runningheadrule

\title{Lineare Algebra 2 \\ \large{Hausaufgabenblatt 09}}
\author{Patrick Gustav Blaneck}
\date{Abgabetermin: 30. Mai 2021}

\begin{document}
\maketitle
\begin{questions}
    \setcounter{question}{4}
    \question
    Finden Sie die Lösung des linearen Gleichungssystems
    $$
        \vektor{-1 & 1 & 0 \\ 0 & 1 & 1}x = \vektor{0 \\ 2}
    $$
    mit Hilfe folgender Schritte:
    \begin{parts}
        \part
        Bestimmen Sie den Kern der Abbildungsmatrix:
        \begin{solution}
            $$
                \begin{sysmatrix}{rrr|r}
                    -1 & 1 & 0 & 0 \\
                    0 & 1 & 1 & 0
                \end{sysmatrix}
                \quad \implies \quad
                \ker \vektor{-1 & 1 & 0 \\ 0 & 1 & 1} = \span{ \left\{\vektor{1 \\ 1 \\ -1}\right\} }
            $$\qed
        \end{solution}

        \part
        Erraten Sie eine spezielle Lösung.
        \begin{solution}
            $$
                x = \vektor{0\\0\\2} \quad \implies \quad \vektor{-1 & 1 & 0 \\ 0 & 1 & 1}\cdot \vektor{0\\0\\2} = \vektor{0 \\ 2}
            $$\qed
        \end{solution}
        \part
        Bestimmen Sie die allgemeine Lösung.
        \begin{solution}
            Es gilt:
            $$
                \Sol \left({\vektor{-1 & 1 & 0 \\ 0 & 1 & 1}, \vektor{0 \\ 2}} \right) = \vektor{0\\0\\2} + \ker\vektor{-1 & 1 & 0 \\ 0 & 1 & 1} = \vektor{0\\0\\2} + \lambda \cdot \vektor{1 \\ 1 \\ -1}, \quad \lambda \in \R
            $$\qed
        \end{solution}
    \end{parts}

    \newpage
    \question
    Gegeben ist das lineare Gleichungssystems $Ax = b$, wobei
    $$
        A = \vektor{1 & 1 & 1 \\ 2 & 1 & 1 \\ 1 & 2 & \lambda}
    $$
    und
    $$
        b = \vektor{3 \\ 5 \\ 2\alpha}
    $$
    Für welche Werte von $\lambda, \alpha \in \R$ existieren
    \begin{parts}
        \part
        keine Lösung,
        \begin{solution}
            $$
                \Sol(A, b) = \emptyset \iff b \notin \span A \iff \rank(A) \neq \rank(A, b)
            $$
            Wir definieren $A'$ wie folgt:
            $$
                A' = \vektor{1 & 1 & 3 \\ 2 & 1 & 5 \\ 1 & 2 & 2\alpha}
            $$

            Dann gilt:
            $$
                \rank(A) \neq \rank(A, b) \iff \abs{A} = 0 \land \abs{A'} \neq 0
            $$
            mit
            $$
                \abs{A} = \lambda + 1 + 4 - 1 - 2\lambda - 2 = 2 - \lambda \implies \abs{A} = 0 \iff \lambda = 2
            $$
            $$
                \abs{A'} = 2\alpha + 5 + 12 - 3 - 10 - 4 \alpha = 4 - 2 \alpha \implies \abs{A'} \neq \iff \alpha \neq 2
            $$
            Damit existiert keine Lösung genau dann, wenn $\lambda = 2$ und $\alpha \neq 2$.\qed
        \end{solution}

        \part
        unendlich viele Lösungen,
        \begin{solution}
            $$
                n = 3 \implies \left(  \left( \abs{A} = 0  \land \rank(A) = \rank(A, b) \right) \implies \abs{\Sol(A, b)} = \infty\right)
            $$

            Wir erkennen, dass $\lambda = 2 \implies \abs{A} = 0$ und $\rank(A) = 2$.

            Da $\vektor{1 & 1 & \lambda}^T = \vektor{1 & 1 & 2}^T$, ist unsere Bedingung erfüllt, wenn $\abs{A'} = 0$ mit:
            $$
                A' = \vektor{1 & 1 & 3 \\ 2 & 1 & 5 \\ 1 & 2 & 2\alpha}
            $$

            $$
                \abs{A'} = 2\alpha + 5 + 12 - 3 - 10 - 4 \alpha = 4 - 2 \alpha \implies \abs{A'} = 0 \iff \alpha = 2
            $$
            Damt existieren unendlich viele Lösungen genau dann, wenn $\lambda = \alpha = 2$.\qed
        \end{solution}

        \newpage
        \part
        eine eindeutige Lösung?
        \begin{solution}
            $$
                n = 3 \implies \left(\abs{A} \neq 0 \implies \rank(A) = \rank(A, b) \implies \abs{\Sol(A, b) = 1}\right)
            $$

            $$
                \abs{A} = \lambda + 1 + 4 - 1 - 2\lambda - 2 = 2 - \lambda \implies \abs{A} \neq 0 \iff \lambda \neq 2
            $$

            Damit existiert eine eindeutige Lösung genau dann, wenn $\lambda \neq 2$ und $\alpha \in \R$.\qed
        \end{solution}
    \end{parts}

    \newpage
    \question
    Anna, Berta und Carla kaufen im Schmuckbedarfsgroßhandel Perlen unterschiedlicher Qualität.

    Es gibt drei verschiedene Sorten.
    Anna kauft jeweils 10 Perlen jeder Sorte und bezahlt 50 Euro.
    Berta nimmt auch 30 Perlen, aber nur von Sorte 1 und 2.
    Sie nimmt jeweils gleich viele und bezahlt 60 Euro.
    Carla bezahlt per Kreditkarte für ihre 60 Perlen 110 Euro.
    Sie wählt 25 von Sorte 1, 25 von Sorte 2 und 10 von Sorte 3.

    Welche Ober- und Untergrenzen für die Einzelpreise ergeben sich?
    \begin{solution}
        Es ist gegeben:
        $$
            \vektor{10 & 10 & 10 \\ 15 & 15 & 0 \\ 25 & 25 & 10} \cdot \vektor{a \\ b \\ c} = \vektor{50 \\ 60 \\ 110}
        $$
        Damit gilt:
        $$
            \begin{sysmatrix}{ccc|c}
                10 & 10 & 10 & 50 \\
                15 & 15 & 0 & 60 \\
                25 & 25 & 10 & 110
            \end{sysmatrix}
            \sim
            \begin{sysmatrix}{ccc|c}
                10 & 10 & 10 & 50 \\
                15 & 15 & 0 & 60 \\
                0 & 0 & 0 & 0
            \end{sysmatrix}
            \sim
            \begin{sysmatrix}{ccc|c}
                15 & 15 & 15 & 75 \\
                15 & 15 & 0 & 60 \\
                0 & 0 & 0 & 0
            \end{sysmatrix}
            \sim
            \begin{sysmatrix}{ccc|c}
                0 & 0 & 1 & 1 \\
                1 & 1 & 0 & 4 \\
                0 & 0 & 0 & 0
            \end{sysmatrix}
        $$

        Damit sieht man direkt, dass
        $$
            a \in \interval{0, 4}, \quad b = 4-a, \quad c = 1.
        $$

        Es gilt für die Grenzen:

        \begin{tabular}{c||c|c}
            Perle & Obergrenze & Untergrenze \\
            \hline
            $a$   & $4$        & $0$         \\
            $b$   & $4$        & $0$         \\
            $c$   & $1$        & $1$
        \end{tabular}
    \end{solution}

    \newpage
    \question
    Welchen Rang haben die Matrizen $A \in \R^{4\times 3}$ und $B \in \R^{n\times 2}$ mit $n\geq 2$?
    \begin{parts}
        \part
        $
            A = \vektor{
                4 & 3 & 6 \\
                -2 & 1 & -8 \\
                1 & 5 & -7 \\
                4 & 2 & -1
            }
        $
        \begin{solution}
            Wir schauen uns die folgende Unterdeterminante an:
            $$
                \dvektor{-2 & 1 & -8 \\ 1 & 5 & -7 \\ 4 & 2 & -1} = 10 -28 -16 + 160 -28 + 1 = 99 \neq 0 \implies \rank(A) = 3
            $$\qed
        \end{solution}

        \part
        $b_{ij} =
            \begin{cases}
                -i,  & i \ \text{gerade}   \\
                i,   & i \ \text{ungerade} \\
                n,   & i = n, j=1          \\
                n-1, & i = n, j=2
            \end{cases}
        $
        \begin{solution}
            $$
                B = \vektor{
                    1 & -2 & 3 & -4 & \ldots & n\\
                    1 & -2 & 3 & -4 & \ldots & n-1
                }^T
            $$
            Wir wissen, dass $\row\rank B = \col\rank B$ und $\rank B \leq 2$.

            Offensichtlich sind $\vektor{1 & 1}$ und $\vektor{n && n-1}$ linear unabhängig.
            Damit ist $\rank B = 2$.\qed
        \end{solution}
    \end{parts}
\end{questions}
\end{document}