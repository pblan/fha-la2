\documentclass[answers]{exam}

\usepackage[ngerman, shorthands=off]{babel}
\usepackage{amsmath,amsthm,amsfonts,stmaryrd,amssymb,mathtools}
\usepackage{xcolor,soul}
\usepackage{polynom}
\usepackage{tikz}
\usetikzlibrary{arrows.meta,angles,quotes,calc}
\usepackage{footnote}
\usepackage{nicefrac}
\usepackage{siunitx}
\usepackage{array}   % for \newcolumntype macro
\usepackage{pgfplots}
\usepgfplotslibrary{fillbetween}
\pgfplotsset{compat=1.8}

\newcolumntype{L}{>{$}l<{$}} % math-mode version of "l" column type
\newcolumntype{R}{>{$}r<{$}} % math-mode version of "r" column type
\newcolumntype{C}{>{$}c<{$}} % math-mode version of "c" column type
\newcolumntype{P}{>{$}p<{$}} % math-mode version of "l" column type

\renewcommand*\env@matrix[1][*\c@MaxMatrixCols c]{%
  \hskip -\arraycolsep
  \let\@ifnextchar\new@ifnextchar
  \array{#1}}

  \newenvironment{sysmatrix}[1]
  {\left(\begin{array}{@{}#1@{}}}
  {\end{array}\right)}

  \newcommand{\Rnum}[1]{\uppercase\expandafter{\romannumeral #1\relax}}
\newcommand{\abs}[1]{\left| #1 \right|}
\newcommand{\cis}[1]{\left( \cos\left( #1 \right) + i \sin\left( #1 \right) \right)}
\newcommand{\sgn}{\text{sgn}} % Signum-Funktion
\newcommand{\diff}{\mathrm{d}} % Differentialquotienten d
\newcommand{\dx}{~\mathrm{d}x} % dx
\newcommand{\du}{~\mathrm{d}u} % du
\newcommand{\dv}{~\mathrm{d}v} % dv
\newcommand{\dw}{~\mathrm{d}w} % dw
\newcommand{\dt}{~\mathrm{d}t} % dt
\newcommand{\dn}{~\mathrm{d}n} % dn
\newcommand{\dudx}{~\frac{\mathrm{d}u}{\mathrm{d}x}} % du/dx
\newcommand{\dudn}{~\frac{\mathrm{d}u}{\mathrm{d}n}} % du/dn
\newcommand{\dvdx}{~\frac{\mathrm{d}v}{\mathrm{d}x}} % dv/dx
\newcommand{\dwdx}{~\frac{\mathrm{d}w}{\mathrm{d}x}} % dw/dx
\newcommand{\dtdx}{~\frac{\mathrm{d}t}{\mathrm{d}x}} % dt/dx
\newcommand{\ddx}{\frac{\mathrm{d}}{\mathrm{d}x}} % d/dx
\newcommand{\dFdx}{\frac{\mathrm{d}F}{\mathrm{d}x}} % dF/dx
\newcommand{\dfdx}{\frac{\mathrm{d}f}{\mathrm{d}x}}  % df/dx
\newcommand{\interval}[1]{\left[ #1 \right]}

\newcommand{\norm}[1]{\left\| #1 \right\|}
\newcommand{\scalarprod}[1]{\left\langle #1 \right\rangle}
\newcommand{\vektor}[1]{\begin{pmatrix*}[c] #1 \end{pmatrix*}}
\newcommand{\dvektor}[1]{\begin{vmatrix*}[c] #1 \end{vmatrix*}}
\renewcommand{\span}[1]{\operatorname{span}\left(#1\right)}

\newcommand{\Nplus}{\mathbb{N}^+}
\newcommand{\N}{\mathbb{N}}
\newcommand{\Z}{\mathbb{Z}}
\newcommand{\Rnonneg}{\mathbb{R}^+_0}
\newcommand{\R}{\mathbb{R}}
\newcommand{\C}{\mathbb{C}}
\newcommand{\bigo}{\mathcal{O}}
\newcommand{\Pot}{\mathcal{P}}
\newcommand{\A}{\mathcal{A}}
\newcommand{\B}{\mathcal{B}}

\DeclareMathOperator{\img}{img}
\DeclareMathOperator{\defect}{defect}
\DeclareMathOperator{\rank}{rank}
\DeclareMathOperator{\trace}{trace}
\DeclareMathOperator{\Sol}{Sol}
\DeclareMathOperator{\row}{row}
\DeclareMathOperator{\col}{col}

\renewcommand{\solutiontitle}{\noindent\textbf{Lösung:}\par}


\makesavenoteenv{solution}
\lhead{Hausaufgabenblatt 11}
\rhead{Lineare Algebra 2}
\runningheadrule

\title{Lineare Algebra 2 \\ \large{Hausaufgabenblatt 11}}
\author{Patrick Gustav Blaneck}
\date{Abgabetermin: 13. Juni 2021}

\begin{document}
\maketitle
\begin{questions}
    \setcounter{question}{4}
    \question
    Die Punkte $A(6;0;0)$, $B(2;1;3)$ und $C(-2;-2;2)$ liegen in einer Ebene $E$.
    \begin{parts}
        \part
        Stellen Sie die Hessesche Normalform der Ebene auf.
        Wie groß ist der Abstand der Ebene zum Ursprung?
        \begin{solution}
            Wir wählen uns $\vec{a}$ (Ortsvektor von $A$) als Stützvektor und die Vektoren $v = \vec{b} - \vec{a}$ und $w = \vec{c} - \vec{a}$ als Richtungsvektoren der Ebene.
            Dann gilt:
            $$
                v = \vektor{2\\1\\3} - \vektor{6\\0\\0} = \vektor{-4\\1\\3}, \qquad w = \vektor{-2\\-2\\2} - \vektor{6\\0\\0} = \vektor{-8\\-2\\2}
            $$
            $$
                n = \frac{v \times w}{\abs{v \times w}} = \frac{1}{\abs{v \times w}} \vektor{8 \\ -16 \\ 16} = \frac{1}{24} \vektor{8 \\ -16 \\ 16} = \vektor{\nicefrac{1}{3} \\ -\nicefrac{2}{3} \\ \nicefrac{2}{3}}
            $$

            Mit $n$ als (normierten) Normalenvektor erhalten wir dann die Hessesche Normalform der Ebene mit
            $$
                E: \scalarprod{x,n} = \scalarprod{\vec{a}, n} \quad \iff \quad \frac{1}{3} \cdot x - \frac{2}{3} \cdot y + \frac{2}{3} \cdot z = 2
            $$

            Setzen wir nun den Nullpunkt in die Ebene ein, erhalten wir sofort den Abstand mit $d = 2$.\qed
        \end{solution}

        \part
        Welcher Punkt in der Ebene hat den kleinsten Abstand zum Ursprung?
        Stellen Sie dazu das zugehörige unterbestimmte LGS auf und finden Sie die Lösung mit Hilfe der verallgemeinerten Inverse.
        \begin{solution}
            Mit der Ebenengleichung
            $$
                E: \frac{1}{3} \cdot x - \frac{2}{3} \cdot y + \frac{2}{3} \cdot z = 2
            $$
            können wir folgendes unterbestimmte LGS aufstellen:
            $$
                Ax = b \quad \iff \quad \vektor{\nicefrac{1}{3} & -\nicefrac{2}{3} & \nicefrac{2}{3}} \vektor{x \\ y \\ z} = \vektor{2}
            $$
            mit
            $$
                \rank(A) = 1 = m \quad \implies \quad x = A^T \left(AA^T\right)^{-1}b
            $$

            Dann gilt:
            $$
                \begin{aligned}
                    x_s \quad = \quad & A^T \left(AA^T\right)^{-1}b \\
                    = \quad           & \vektor{\nicefrac{1}{3}     \\ -\nicefrac{2}{3} \\ \nicefrac{2}{3}} \left( \vektor{\nicefrac{1}{3} & -\nicefrac{2}{3} & \nicefrac{2}{3}} \vektor{\nicefrac{1}{3} \\ -\nicefrac{2}{3} \\ \nicefrac{2}{3}} \right)^{-1} \cdot 2 \\
                    = \quad           & \vektor{\nicefrac{1}{3}     \\ -\nicefrac{2}{3} \\ \nicefrac{2}{3}} \cdot 1^{-1} \cdot 2 \\
                    = \quad           & \vektor{\nicefrac{1}{3}     \\ -\nicefrac{2}{3} \\ \nicefrac{2}{3}} \cdot 1 \cdot 2 \\
                    = \quad           & \vektor{\nicefrac{2}{3}     \\ -\nicefrac{4}{3} \\ \nicefrac{4}{3}} = \vektor{x\\y\\z}
                \end{aligned}
            $$
            Damit ist dann $\vektor{\nicefrac{2}{3} & -\nicefrac{4}{3} & \nicefrac{4}{3}}^T$ der gesuchte Punkt in der Ebene mit dem geringsten Abstand.\qed
        \end{solution}
    \end{parts}

    \newpage
    \question
    Zeigen Sie, dass die Matrix
    $$
        Q = \vektor{
            \cos \beta & -\sin\beta & 0 \\
            \cos \alpha \sin\beta & \cos\alpha \cos\beta & -\sin\alpha \\
            \sin\alpha \sin\beta & \sin\alpha \cos\beta & \cos\alpha
        }
    $$
    eine Orthogonalmatrix ist und bestimmen Sie ihre Inverse.
    \begin{solution}
        Genau dann, wenn $Q$ eine Orthogonalmatrix ist, ist $QQ^T = I$ und damit auch $Q^T = Q^{-1}$:
        $$
            \begin{aligned}
                QQ^T \quad = \quad  &
                \vektor{ \cos \beta & -\sin\beta                        & 0                        \\ \cos \alpha \sin\beta & \cos\alpha \cos\beta & -\sin\alpha \\ \sin\alpha \sin\beta & \sin\alpha \cos\beta & \cos\alpha }
                \vektor{ \cos\beta  & \cos\alpha \sin\beta              & \sin\alpha \sin\beta     \\ -\sin\beta & \cos\alpha \cos\beta & \sin\alpha \cos\beta \\ 0 & -\sin\alpha & \cos\alpha} \\
                = \quad             & \vektor{\cos^2\beta + \sin^2\beta & 0                    & 0 \\ 0 & \cos^2\alpha \left( \sin^2\beta + \cos^2\beta \right) + \sin^2\alpha & 0  \\ 0 & 0 & \sin^2\alpha \left( \sin^2\beta + \cos^2\beta \right) + \cos^2\alpha} \\
                = \quad             & \vektor{\cos^2\beta + \sin^2\beta & 0                    & 0 \\ 0 & \cos^2\alpha  + \sin^2\alpha & 0  \\ 0 & 0 & \sin^2\alpha + \cos^2\alpha} \\
                = \quad             & \vektor{1                         & 0                    & 0 \\ 0 & 1 & 0  \\ 0 & 0 & 1}
            \end{aligned}
        $$

        Damit ist $Q$ eine Orthogonalmatrix und $Q^T$ die Inverse von $Q$.\qed
    \end{solution}

    \newpage
    \question
    Die Abbildung $f_A$ dreht einen Vektor im $\R^3$ innerhalb der $x$-$z$-Ebene um einen Winkel $\phi$.
    Die Abbildung $f_B$ spiegelt einen Vektor im $\R^3$ an der $x$-Achse.
    \begin{parts}
        \part
        Stellen Sie die zugehörigen Abbildungsmatrizen $A$ und $B$ auf.
        \begin{solution}
            Es gilt:
            $$
                A = \vektor{\cos\phi & 0 & -\sin\phi \\ 0 & 1 & 0 \\ \sin\phi & 0 & \cos\phi}, \qquad B = \vektor{1 & 0 & 0 \\ 0 & -1 & 0 \\ 0 & 0 & -1}
            $$\qed
        \end{solution}

        \part
        Stellen Sie die zugehörige Abbildungsmatrix der hintereinander geschalteten Abbildungen $f_B \circ f_A$ auf.
        \begin{solution}
            Die Abbildungsmatrix von $f_B \circ f_A$ ist gegeben mit:
            $$
                M = BA = \vektor{1 & 0 & 0 \\ 0 & -1 & 0 \\ 0 & 0 & -1} \vektor{\cos\phi & 0 & -\sin\phi \\ 0 & 1 & 0 \\ \sin\phi & 0 & \cos\phi} = \vektor{\cos\phi & 0 & -\sin\phi \\ 0 & -1 & 0 \\ -\sin\phi & 0 & -\cos\phi }
            $$\qed
        \end{solution}

        \part Bestimmen Sie auch die zugehörige Abbildungsmatrix der Umkehrabbildung $\left( f_B \circ f_A \right)^{-1}$.
        \begin{solution}
            Wir erkennen sehr schnell, dass $M$ eine Orthogonalmatrix ist (und insbesondere $M = M^T$).
            Damit gilt dann:
            $$
                \begin{aligned}
                    MM^T \quad = \quad & \vektor{\cos\phi                & 0 & -\sin\phi \\ 0 & -1 & 0 \\ -\sin\phi & 0 & -\cos\phi } \vektor{\cos\phi & 0 & -\sin\phi \\ 0 & -1 & 0 \\ -\sin\phi & 0 & -\cos\phi } \\
                    = \quad            & \vektor{\cos^2\phi + \sin^2\phi & 0 & 0         \\ 0 & 1 & 0  \\ 0 & 0 & \sin^2\phi + \cos^2\phi } \\
                    = \quad            & \vektor{1                       & 0 & 0         \\ 0 & 1 & 0  \\ 0 & 0 & 1}
                \end{aligned}
            $$

            Damit ist $M$ also insgesamt involutorisch (selbstinvers). \qed
        \end{solution}
    \end{parts}

    \newpage
    \question
    \begin{parts}
        \part Zeigen Sie, dass die symmetrische Matrix $H_n$ für jeden Spaltenvektor $u \in \R^n \setminus \{0\}$ orthogonal ist:
        $$
            H_n := I_n - 2 \cdot \frac{uu^T}{u^Tu}
        $$
        $I_n$ ist dabei die $(n\times n)$-Einheitsmatrix.

        \emph{Hinweis:} Berechnen Sie nicht die Komponenten von $H_n$.
        \begin{solution}
            Es gilt offensichtlich:
            $$
                \begin{aligned}
                    H_n := \quad                                                                                                & I_n - 2 \cdot \frac{uu^T}{u^Tu} \quad \overset{\footnotemark[1]}{=} \footnotetext[1]{$u \in \R^{n\times 1} \implies \left(uu^T\in \R^{n\times n}\right) \land \left(u^Tu  = \scalarprod{u, u}\in \R\right)$} \quad  I_n - 2\cdot \frac{uu^T}{\scalarprod{u, u}} \\
                    \implies \footnotetext[2]{$H_n = (H_n)^T$} \quad H_n \left(H_n\right)^T \overset{\footnotemark[2]}{=} \quad & \left( I_n - 2\cdot \frac{uu^T}{\scalarprod{u, u}} \right)^2                                                                                                                                                                                                    \\
                    = \quad                                                                                                     & \left( I_n - 2\cdot \frac{uu^T}{\scalarprod{u, u}} \right) \left( I_n - 2\cdot \frac{uu^T}{\scalarprod{u, u}} \right)                                                                                                                                           \\
                    = \quad                                                                                                     & I_n - 2\cdot \frac{uu^T}{\scalarprod{u, u}}  - 2\left( \frac{uu^T}{\scalarprod{u, u}} \right) \left( I_n - 2\cdot \frac{uu^T}{\scalarprod{u, u}} \right)                                                                                                        \\
                    = \quad                                                                                                     & I_n - 2\cdot \frac{uu^T}{\scalarprod{u, u}} - 2 \cdot \frac{uu^T}{\scalarprod{u, u}} + \left(-2 \cdot  \frac{uu^T}{\scalarprod{u, u}} \right)^2                                                                                                                 \\
                    = \quad                                                                                                     & I_n - 4\cdot \frac{uu^T}{\scalarprod{u, u}} + 4 \left( \frac{uu^T}{\scalarprod{u, u}} \right)^2                                                                                                                                                                 \\
                    = \quad                                                                                                     & I_n - 4\cdot \frac{uu^T}{\scalarprod{u, u}} + 4 \cdot \frac{uu^Tuu^T}{\scalarprod{u, u}^2}                                                                                                                                                                      \\
                    = \quad                                                                                                     & I_n - 4\cdot \frac{uu^T}{\scalarprod{u, u}} + 4 \cdot \frac{u\left(u^Tu\right)u^T}{\scalarprod{u, u}^2}                                                                                                                                                         \\
                    = \quad                                                                                                     & I_n - 4\cdot \frac{uu^T}{\scalarprod{u, u}} + 4 \cdot \frac{uu^T}{\scalarprod{u, u}} \quad = \quad I_n                                                                                                                                                          \\
                \end{aligned}
            $$

            Damit ist $H_n \left(H_n\right)^T = I_n$ und nach Definition $H_n$ eine Orthogonalmatrix. \qed
        \end{solution}

        \part
        Verifizieren Sie das Ergebnis für $u = \vektor{1\\0\\2}$.
        \begin{solution}
            Es gilt:
            $$
                I_3 - 2 \cdot \frac{uu^T}{u^Tu} = I_3 - \frac{2}{5} \vektor{1 & 0 & 2 \\ 0 & 0 & 0 \\ 2 & 0 & 4} = \vektor{\nicefrac{3}{5} & 0 & -\nicefrac{4}{5} \\ 0 & 1 & 0 \\ -\nicefrac{4}{5} & 0 & -\nicefrac{3}{5}} \in O(3)
            $$\qed
        \end{solution}
    \end{parts}
\end{questions}
\end{document}