\documentclass[answers]{exam}

\usepackage[ngerman, shorthands=off]{babel}
\usepackage{amsmath,amsthm,amsfonts,stmaryrd,amssymb,mathtools}
\usepackage{xcolor,soul}
\usepackage{polynom}
\usepackage{tikz}
\usetikzlibrary{arrows.meta,angles,quotes,calc}
\usepackage{footnote}
\usepackage{nicefrac}
\usepackage{siunitx}
\usepackage{array}   % for \newcolumntype macro
\usepackage{pgfplots}
\usepgfplotslibrary{fillbetween}
\pgfplotsset{compat=1.8}

\newcolumntype{L}{>{$}l<{$}} % math-mode version of "l" column type
\newcolumntype{R}{>{$}r<{$}} % math-mode version of "r" column type
\newcolumntype{C}{>{$}c<{$}} % math-mode version of "c" column type
\newcolumntype{P}{>{$}p<{$}} % math-mode version of "l" column type

\renewcommand*\env@matrix[1][*\c@MaxMatrixCols c]{%
  \hskip -\arraycolsep
  \let\@ifnextchar\new@ifnextchar
  \array{#1}}

  \newenvironment{sysmatrix}[1]
  {\left(\begin{array}{@{}#1@{}}}
  {\end{array}\right)}

  \newcommand{\Rnum}[1]{\uppercase\expandafter{\romannumeral #1\relax}}
\newcommand{\abs}[1]{\left| #1 \right|}
\newcommand{\cis}[1]{\left( \cos\left( #1 \right) + i \sin\left( #1 \right) \right)}
\newcommand{\sgn}{\text{sgn}} % Signum-Funktion
\newcommand{\diff}{\mathrm{d}} % Differentialquotienten d
\newcommand{\dx}{~\mathrm{d}x} % dx
\newcommand{\du}{~\mathrm{d}u} % du
\newcommand{\dv}{~\mathrm{d}v} % dv
\newcommand{\dw}{~\mathrm{d}w} % dw
\newcommand{\dt}{~\mathrm{d}t} % dt
\newcommand{\dn}{~\mathrm{d}n} % dn
\newcommand{\dudx}{~\frac{\mathrm{d}u}{\mathrm{d}x}} % du/dx
\newcommand{\dudn}{~\frac{\mathrm{d}u}{\mathrm{d}n}} % du/dn
\newcommand{\dvdx}{~\frac{\mathrm{d}v}{\mathrm{d}x}} % dv/dx
\newcommand{\dwdx}{~\frac{\mathrm{d}w}{\mathrm{d}x}} % dw/dx
\newcommand{\dtdx}{~\frac{\mathrm{d}t}{\mathrm{d}x}} % dt/dx
\newcommand{\ddx}{\frac{\mathrm{d}}{\mathrm{d}x}} % d/dx
\newcommand{\dFdx}{\frac{\mathrm{d}F}{\mathrm{d}x}} % dF/dx
\newcommand{\dfdx}{\frac{\mathrm{d}f}{\mathrm{d}x}}  % df/dx
\newcommand{\interval}[1]{\left[ #1 \right]}

\newcommand{\norm}[1]{\left\| #1 \right\|}
\newcommand{\scalarprod}[1]{\left\langle #1 \right\rangle}
\newcommand{\vektor}[1]{\begin{pmatrix*}[c] #1 \end{pmatrix*}}
\renewcommand{\span}[1]{\operatorname{span}\left(#1\right)}

\newcommand{\Nplus}{\mathbb{N}^+}
\newcommand{\N}{\mathbb{N}}
\newcommand{\Z}{\mathbb{Z}}
\newcommand{\Rnonneg}{\mathbb{R}^+_0}
\newcommand{\R}{\mathbb{R}}
\newcommand{\C}{\mathbb{C}}
\newcommand{\bigo}{\mathcal{O}}
\newcommand{\Pot}{\mathcal{P}}
\newcommand{\A}{\mathcal{A}}
\newcommand{\B}{\mathcal{B}}

\DeclareMathOperator{\img}{img}
\DeclareMathOperator{\defect}{defect}
\DeclareMathOperator{\rank}{rank}
\DeclareMathOperator{\trace}{trace}

\renewcommand{\solutiontitle}{\noindent\textbf{Lösung:}\par}


\makesavenoteenv{solution}
\lhead{Hausaufgabenblatt 07}
\rhead{Lineare Algebra 2}
\runningheadrule

\title{Lineare Algebra 2 \\ \large{Hausaufgabenblatt 07}}
\author{Patrick Gustav Blaneck}
\date{Abgabetermin: 16. Mai 2021}

\begin{document}
\maketitle
\begin{questions}
    \setcounter{question}{4}
    \question
    Ein Architekt plant, auf dem Dach eines Hauses eine Antenne anzubringen (siehe Skizze).
    Von seinem Bezugspunkt $A$ aus gesehen, soll sie senkrecht über der Stelle, die auf der Grundfläche des Hauses 5m nach rechts ($x$-Richtung) und 2m nach hinten ($y$-Richtung) liegt, auf dem Dach angebracht werden.

    \begin{center}
        \begin{tikzpicture}[scale=1]
            \begin{axis}[
                    view={45}{15},
                    width=20cm,
                    unit vector ratio*=1 1 1,
                    axis lines = middle,
                    ymin=-1,
                    ymax=7,
                    xmin=-1,
                    xmax=12,
                    zmin=-1,
                    zmax=10,
                    xlabel = $x$,
                    ylabel = $y$,
                    zlabel = $z$,
                    xtick style={draw=none},
                    ytick style={draw=none},
                    ztick style={draw=none},
                    xticklabels=\empty,
                    yticklabels=\empty,
                    zticklabels=\empty,
                    disabledatascaling,
                ]

                \node (A) at (axis cs:0,0,0) {} node[below left] {$A$};
                \node (B) at (axis cs:10,0,0) {};
                \node (C) at (axis cs:10,5,0) {};
                \node (D) at (axis cs:0,5,0) {};
                \node (E) at (axis cs:0,0,6) {};
                \node (F) at (axis cs:10,0,6) {} node[below right] {$B$};
                \node (G) at (axis cs:10,5,6) {};
                \node (H) at (axis cs:0,5,6) {};
                \node (J) at (axis cs:0,2.5,8.5) {};
                \node (I) at (axis cs:10,2.5,8.5) {};


                \draw[] (A.center) -- (B.center) node [midway, below] {\small{10m}};
                \draw[] (B.center) -- (C.center) node [midway, below right] {\small{5m}};
                \draw[] (A.center) -- (E.center) node [midway, left] {\small{6m}};
                \draw[] (B.center) -- (F.center);
                \draw[] (C.center) -- (G.center);
                \draw[] (E.center) -- (F.center);
                \draw[] (F.center) -- (G.center);
                \draw[] (E.center) -- (J.center);
                \draw[] (G.center) -- (I.center);
                \draw[] (F.center) -- (I.center);
                \draw[] (I.center) -- (G.center);
                \draw[] (I.center) -- (J.center);

                \draw[dashed] (C.center) -- (D.center);
                \draw[dashed] (A.center) -- (D.center);
                \draw[dashed] (E.center) -- (H.center);
                \draw[dashed] (D.center) -- (H.center);
                \draw[dashed] (G.center) -- (H.center);
                \draw[dashed] (J.center) -- (H.center);

                \draw[->, red] (F.center) -- (axis cs:8,0,6) node [midway, below left] {$x'$};
                \draw[->, red] (F.center) -- (axis cs:10,1,6) node [midway, below right] {$y'$};
                \draw[->, red] (F.center) -- (axis cs:10,1,7) node [midway, left] {$z'$};

                \pic [draw, "\tiny{$45^\circ$}", angle radius=20] {angle = G--F--I };
                \pic [draw, "\tiny{$90^\circ$}", angle radius=20] {angle = F--I--G };

                \draw[] (axis cs:5,2,8) -- (axis cs:5,2,9.2);
                \draw[] (axis cs:4.5,2,9) -- (axis cs:5.5,2,9);
                \draw[] (axis cs:4.5,2,8.9) -- (axis cs:5.5,2,8.9);
            \end{axis}

        \end{tikzpicture}
    \end{center}

    \newpage
    \begin{parts}
        \part
        Berechnen Sie den Anfangspunkt der Antenne auf dem Dach vom Bezugspunkt $A$ aus gesehen.
        \begin{solution}
            Wir drehen uns einmal unsere Darstellung des Hauses wie folgt:

            \begin{center}
                \begin{tikzpicture}[scale=0.75]
                    \begin{axis}[
                            view={85}{5},
                            width=20cm,
                            unit vector ratio*=1 1 1,
                            axis lines = middle,
                            ymin=-1,
                            ymax=7,
                            xmin=-1,
                            xmax=12,
                            zmin=-1,
                            zmax=10,
                            xlabel = $x$,
                            ylabel = $y$,
                            zlabel = $z$,
                            xtick style={draw=none},
                            ytick style={draw=none},
                            ztick style={draw=none},
                            xticklabels=\empty,
                            yticklabels=\empty,
                            zticklabels=\empty,
                            disabledatascaling,
                        ]

                        \node (A) at (axis cs:0,0,0) {} node[below left] {$A$};
                        \node (B) at (axis cs:10,0,0) {};
                        \node (C) at (axis cs:10,5,0) {};
                        \node (D) at (axis cs:0,5,0) {};
                        \node (E) at (axis cs:0,0,6) {};
                        \node (F) at (axis cs:10,0,6) {};
                        \node (G) at (axis cs:10,5,6) {};
                        \node (H) at (axis cs:0,5,6) {};
                        \node (J) at (axis cs:0,2.5,8.5) {};
                        \node (I) at (axis cs:10,2.5,8.5) {};

                        \draw[] (A.center) -- (B.center);
                        \draw[] (B.center) -- (C.center);
                        \draw[] (A.center) -- (E.center);
                        \draw[] (B.center) -- (F.center);
                        \draw[] (C.center) -- (G.center);
                        \draw[] (E.center) -- (F.center);
                        \draw[] (F.center) -- (G.center);
                        \draw[] (E.center) -- (J.center);
                        \draw[] (G.center) -- (I.center);
                        \draw[] (F.center) -- (I.center);
                        \draw[] (I.center) -- (G.center);
                        \draw[] (I.center) -- (J.center);

                        \draw[dashed] (C.center) -- (D.center);
                        \draw[dashed] (A.center) -- (D.center);
                        \draw[dashed] (E.center) -- (H.center);
                        \draw[dashed] (D.center) -- (H.center);
                        \draw[dashed] (G.center) -- (H.center);
                        \draw[dashed] (J.center) -- (H.center);

                        \node (Z1) at (axis cs:5,2,6) {};
                        \node (Z2) at (axis cs:0,2,6) {};
                        \node (Z3) at (axis cs:5,2,8) {};

                        \draw[dotted] (axis cs:0,2,0) -- (axis cs:5,2,0);
                        \draw[dotted] (axis cs:5,0,0) -- (axis cs:5,2,0) -- (axis cs:5,2,6);
                        \draw[red, dashed] (axis cs:5,2,6) -- (axis cs:5,2,8) node [midway, right] {$z'$};

                        \draw[red, dotted] (axis cs:5,2,6) -- (axis cs:5,0,6) -- (axis cs:5,2,8);

                        \draw[fill=gray, fill opacity=0.15] (E.center) -- (F.center) -- (G.center) -- (H.center) -- (E.center) -- cycle;

                        \draw[] (axis cs:5,2,8) -- (axis cs:5,2,9.2);
                        \draw[] (axis cs:4.5,2,9) -- (axis cs:5.5,2,9);
                        \draw[] (axis cs:4.5,2,8.9) -- (axis cs:5.5,2,8.9);
                    \end{axis}

                \end{tikzpicture}
            \end{center}

            Die $x$- bzw. $y$-Koordinaten des Antennenfußes sind aus der Aufgabe bereits gegeben mit $C_x = 5$ und $C_y = 2$.

            Aus der Zeichnung wird weiterhin ersichtlich, dass sich die $z$-Koordinate zum einen aus der Höhe des Hauses, und zum anderen aus der gekennzeichneten Strecke $z'$ zusammensetzt.
            Es gilt:
            $$
                C_z = 6 + z' = 6 + \tan(45^\circ) \cdot 2 = 6 + 2\tan\left(\frac{\pi}{4}\right) = 8
            $$

            Damit befindet sich der Anfangspunkt der Antenne bei $C = \vektor{5 & 2 & 8}^T$.\qed
        \end{solution}

        \newpage
        \part
        Der Dachdecker, der an dieser Stelle Dachziegel weglassen muss, nimmt als Bezugssystem die rechte untere Ecke des Daches $B$ und als Basisvektoren die eingezeichneten Richtungsvektoren $x'$, $y'$ und $z'$ (der Länge 1).
        Berechnen Sie die Koordinaten des Punktes $C$ bzgl. seines Koordinatensystems.
        \begin{solution}
            Sei $\mathcal{A}$ die kanonische Einheitsbasis des $\R^3$ und $\mathcal{D}$ die Dachdeckerbasis.

            Dann wissen wir, dass unsere Basis zuerst wie folgt verändert wird:
            $$
                T^{\mathcal{A}}_{\mathcal{D}} = \mathcal{D}^{-1} \cdot \mathcal{A} = \vektor{-1 & 0 & 0 \\ 0 & 1 & \nicefrac{1}{\sqrt{2}} \\ 0 & 0 & \nicefrac{1}{\sqrt{2}}}^{-1} \cdot I = \vektor{-1 & 0 & 0 \\ 0 & 1 & \nicefrac{1}{\sqrt{2}} \\ 0 & 0 & \nicefrac{1}{\sqrt{2}}}^{-1}.
            $$

            Es gilt:
            $$
                \begin{sysmatrix}{rrr|rrr}
                    -1 & 0 & 0 & 1 & 0 & 0 \\
                    0 & 1 & \nicefrac{1}{\sqrt{2}} & 0 & 1 & 0 \\
                    0 & 0 & \nicefrac{1}{\sqrt{2}} & 0 & 0 & 1
                \end{sysmatrix}
                \sim
                \begin{sysmatrix}{rrr|rrr}
                    -1 & 0 & 0 & 1 & 0 & 0 \\
                    0 & 1 & \nicefrac{1}{\sqrt{2}} & 0 & 1 & 0 \\
                    0 & 0 & 1 & 0 & 0 & \sqrt{2}
                \end{sysmatrix}
                \sim
                \begin{sysmatrix}{rrr|rrr}
                    1 & 0 & 0 & -1 & 0 & 0 \\
                    0 & 1 & 0 & 0 & 1 & -1 \\
                    0 & 0 & 1 & 0 & 0 & \sqrt{2}
                \end{sysmatrix}
            $$
            und damit:
            $$
                T^{\mathcal{A}}_{\mathcal{D}} = \vektor{-1 & 0 & 0 \\ 0 & 1 & -1 \\ 0 & 0 & \sqrt{2}}.
            $$

            Möchten wir nun den Punkt $B$ in unserer Basis $\mathcal{D}$ darstellen, gilt:
            $$
                B = K_{\mathcal{D}}(B) = T^{\mathcal{A}}_{\mathcal{D}} \cdot B = \vektor{-1 & 0 & 0 \\ 0 & 1 & -1 \\ 0 & 0 & \sqrt{2}} \cdot \vektor{10\\0\\6} = \vektor{-10 \\ -6 \\ 6\sqrt{2}}.
            $$

            Damit gilt, dass unser Koordinatensystem wie folgt verschoben wird:
            $$
                T(x', y', z') \to (x'+10, y'+6, z' - 6\sqrt{2}).
            $$

            Stellen wir nun $C$ in Basis $\mathcal{D}$ dar und wenden danach die beschriebene Translation an, erhalten wir:
            $$
                C = K_{\mathcal{D}}(C) = T^{\mathcal{A}}_{\mathcal{D}} \cdot C = \vektor{-1 & 0 & 0 \\ 0 & 1 & -1 \\ 0 & 0 & \sqrt{2}} \cdot \vektor{5\\2\\8} = \vektor{-5 \\ -6 \\ 8\sqrt{2}} \quad \text{ und }
            $$
            $$
                T(-5, -6, 8\sqrt{2}) \to (5, 0, 2\sqrt{2}).
            $$\qed
        \end{solution}

        \part
        Wie lauten die Koordinaten des Bezugpunktes $A$ des Architekten im Koordinatensystem des Dachdeckers?
        \begin{solution}
            Wieder müssen wir zuerst eine Koordinatentransformation und anschließend eine Translation anwenden:
            $$
                A = K_{\mathcal{D}} (A) = T^{\mathcal{A}}_{\mathcal{D}} \cdot A = \vektor{0\\0\\0} \quad \text{und} \quad T(0,0,0) \to (10, 6, -6\sqrt{2})
            $$\qed
        \end{solution}

        \newpage
        \part
        Wie muss die Transformation (Matrix und Verschiebungsvektor) aussehen, die einen beliebigen Punkt des Hauses aus dem Koordinatensystem des Dachdeckers in das des Architekten umrechnet?
        Überprüfen Sie das Ergebnis, indem Sie das Ergebnis von (b) in das Ergebnis von (a) umrechnen.
        \begin{solution}
            Siehe Aufgabenteil (b).
        \end{solution}

        \part
        Wie muss die Transformation (Matrix und Verschiebungsvektor) aussehen, die einen beliebigen Punkt des Hauses aus dem Koordinatensystem des Architekten in das des Dachdeckers umrechnet?
        \begin{solution}
            Für die Basistransformation gilt:
            $$
                T^{\mathcal{A}}_{\mathcal{D}} = \left(T^{\mathcal{A}}_{\mathcal{D}} \right)^{-1} = \mathcal{D} = \vektor{-1 & 0 & 0 \\ 0 & 1 & \nicefrac{1}{\sqrt{2}} \\ 0 & 0 & \nicefrac{1}{\sqrt{2}}}
            $$

            Für die Translation gilt:
            $$
                T(x, y, z) \to (x + 10, y, z + 6)\footnote{$T(x) = x -(- B)$}
            $$\qed
        \end{solution}
    \end{parts}

    \newpage
    \question
    Die Determinante der folgenden Matrix $A_1$ ist:
    $$
        \det(A_1) = \det \vektor{a_{11} & a_{12} & a_{13} & a_{14} \\ a_{21} & a_{22} & a_{23} & a_{24} \\ a_{31} & a_{32} & a_{33} & a_{34} \\ a_{41} & a_{42} & a_{43} & a_{44}} = d
    $$
    Berechnen Sie mit Hilfe von $\det(A_1)$ die Determinanten der folgenden Matrizen in Abhängigkeit von $c, d \in \R$:
    \begin{parts}
        \part
        $$
            A_2 = \vektor{a_{11} & a_{12} & a_{13} & a_{14} \\ a_{21} & a_{22} & a_{23} & a_{24} \\ c\cdot a_{31} & c\cdot a_{32} & c\cdot a_{33} & c\cdot a_{34} \\ a_{41} & a_{42} & a_{43} & a_{44}}
        $$
        \begin{solution}
            $$
                A_2 = \vektor{1 & 0 & 0 & 0 \\ 0 & 1 & 0 & 0 \\ 0 & 0 & c & 0 \\ 0 & 0 & 0 & 1} \cdot A_1 \implies \abs{A_2} = c \cdot \abs{A_1} = cd
            $$\qed
        \end{solution}

        \part
        $$
            A_3 =c\cdot  \vektor{a_{11} + a_{41} & a_{12} + a_{43}  & a_{13} + a_{43}  & a_{14} + a_{44}  \\ a_{21} & a_{22} & a_{23} & a_{24} \\ a_{31} & a_{32} & a_{33} & a_{34} \\ a_{41} & a_{42} & a_{43} & a_{44}}
        $$
        \begin{solution}
            $$
                A_3 = \vektor{1 & 0 & 0 & 1 \\ 0 & 1 & 0 & 0 \\ 0 & 0 & 1 & 0 \\ 0 & 0 & 0 & 1} \cdot c\cdot A_1 \implies \abs{A_3} =\footnote{$\forall M \in K^{n\times n} : \det(\lambda M) = \lambda^n \cdot \det(M)$, da jede Zeile mit $\lambda$ skaliert wird.} 1 \cdot c^4\cdot  \abs{A_1} = c^4d
            $$\qed
        \end{solution}

        \newpage
        \part
        $$
            A_4 = \vektor{a_{12} & a_{11} & a_{13} & a_{14} \\ a_{22} & a_{21} & a_{23} & a_{24} \\ a_{32} & a_{31} & a_{33} & a_{34} \\ a_{42} & a_{41} & a_{43} & a_{44}}
        $$
        \begin{solution}
            $$
                A_4 = A_1 \cdot \vektor{0 & 1 & 0 & 0 \\ 1 & 0 & 0 & 0 \\ 0 & 0 & 1 & 0 \\ 0 & 0 & 0 & 1}\implies \abs{A_4} = \abs{A_1}\cdot (-1) = -d
            $$\qed
        \end{solution}
        \part
        $$
            A_5 = \vektor{a_{11} + c\cdot a_{12} & a_{12} & a_{13} & a_{14} \\ a_{21} + c\cdot a_{22} & a_{22} & a_{23} & a_{24} \\ a_{31} + c\cdot a_{32} & a_{32} & a_{33} & a_{34} \\ a_{41} + c\cdot a_{42} & a_{42} & a_{43} & a_{44}}
        $$
        \begin{solution}
            $$
                A_5 = A_1 \cdot \vektor{1 & 0 & 0 & 0 \\ c & 1 & 0 & 0 \\ 0 & 0 & 1 & 0 \\ 0 & 0 & 0 & 1}\implies \abs{A_5} = \abs{A_1}\cdot 1 = d
            $$\qed
        \end{solution}
        \part
        $$
            A_6 = \vektor{a_{11} & a_{12} & a_{13} & a_{14} \\ a_{21} & a_{22} & a_{23} & a_{24} \\ a_{11} + c\cdot a_{31} & a_{12} + c\cdot a_{32} & a_{13} + c\cdot a_{33} & a_{14} + c\cdot a_{34} \\ a_{41} & a_{42} & a_{43} & a_{44}}
        $$
        \begin{solution}
            $$
                A_6 = \vektor{1 & 0 & 0 & 0 \\ 0 & 1 & 0 & 0 \\ 1 & 0 & 1 & 0 \\ 0 & 0 & 0 & 1} \cdot \vektor{1 & 0 & 0 & 0 \\ 0 & 1 & 0 & 0 \\ 0 & 0 & c & 0 \\ 0 & 0 & 0 & 1} \cdot A_1 \implies \abs{A_6} = 1 \cdot c \cdot \abs{A_1} = cd
            $$\qed
        \end{solution}

        \newpage
        \part
        $$
            A_7 = (A_1)^{-1}
        $$
        \begin{solution}
            Wir wissen, dass wir mittels elementarer Zeilenoperationen (äquivalent zur linksseitigen Multiplikation mit Elementarmatrizen) $A_1$ zur Einheitsmatrix umformen können.

            Sei $Z^{(i)}$ eine beliebige Elementarmatrix\footnote{Hier betrachten wir nur Zeilenoperationen.}. Damit gilt:
            $$
                I = \underbrace{\left(Z^{(n)} \cdot Z^{(n-1)} \cdot \ldots \cdot Z^{(2)} \cdot Z^{(1)}\right)}_{(A_1)^{-1}} \cdot A_1 = (A_1)^{-1} \cdot A_1
            $$
            $$
                \implies \abs{I} = \abs{(A_1)^{-1}\cdot A_1} = \abs{(A_1)^{-1}} \cdot \abs{A_1}
            $$
            $$
                \iff \abs{(A_1)^{-1}} = \frac{1}{\abs{A_1}} \implies \abs{(A_1)^{-1}} = \frac{1}{d}
            $$\qed
        \end{solution}
    \end{parts}

    \newpage
    \question
    Sei
    $$
        A = \vektor{1 & 0 \\ -5 & 2}
    $$
    \begin{parts}
        \part
        Bestimmen Sie Elementarmatrizen $M_1$ und $M_2$ mit $M_1M_2A = E$.
        \begin{solution}
            Es gilt:
            $$
                \begin{sysmatrix}{cc|cc}
                    1 & 0 & 1 & 0 \\
                    -5 & 2 & 0 & 1
                \end{sysmatrix}
                \sim
                \begin{sysmatrix}{cc|cc}
                    1 & 0 & 1 & 0 \\
                    0 & 2 & 5 & 1
                \end{sysmatrix}
                \sim
                \begin{sysmatrix}{cc|cc}
                    1 & 0 & 1 & 0 \\
                    0 & 1 & \frac{5}{2} & \frac{1}{2}
                \end{sysmatrix}
            $$
            Damit gilt:
            $$
                A^{-1} = \vektor{1 & 0 \\ \frac{5}{2} & \frac{1}{2}} =\footnote{Das wird aus dem Kontext ersichtlich: Für $A^{-1}$ wird zuerst fünfmal \Rnum{1} auf \Rnum{2} addiert und anschließend wird \Rnum{2} mit $\frac{1}{2}$ skaliert.} \vektor{1 & 0 \\ 0 & \frac{1}{2}} \cdot \vektor{1 & 0 \\ 5 & 1}
            $$
            und schlussendlich:
            $$
                A^{-1} \cdot A = \vektor{1 & 0 \\ 0 & \frac{1}{2}} \cdot \vektor{1 & 0 \\ 5 & 1} \cdot A = M_1 M_2  A
            $$\qed
        \end{solution}

        \part
        Stellen Sie $A^{-1}$ als Produkt zweier Elementarmatrizen dar.
        \begin{solution}
            Siehe Aufgabenteil (a).
        \end{solution}

        \part
        Schreiben Sie $A$ als Produkt von zwei Elementarmatrizen.
        \begin{solution}
            Es gilt:
            $$
                A = \vektor{1 & 0 \\ -5 & 2} =\footnote{Analoge Argumentation zu $A^{-1}$.} \vektor{1 & 0 \\ -5 & 1} \cdot \vektor{1 & 0 \\ 0 & 2}
            $$\qed
        \end{solution}
    \end{parts}

    \newpage
    \question
    Gegeben seien drei Matrizen $A, B, C \in \R^{2\times 2}$ mit folgenden Eigenschaften:
    \begin{itemize}
        \item $A$ ist regulär mit $\det(A) = 10$.
        \item Für $B$ gilt: $\det(A\cdot B) = 1$.
        \item Für $C$ gilt: $B\cdot C$ ist singulär, d.h. nicht invertierbar.
    \end{itemize}
    berechnen Sie:
    \begin{parts}
        \part
        $\det(C \cdot A)$
        \begin{solution}
            Wir wissen:
            $$
                B \cdot C \ \text{singulär} \iff \det(B\cdot C) = 0 \iff B \ \text{nicht invertierbar} \lor C \ \text{nicht invertierbar}
            $$
            und
            $$
                \det(A\cdot B) = 1 \neq 0 \implies A \ \text{invertierbar} \land B \ \text{invertierbar}
            $$
            Damit gilt dann schließlich:
            $$
                A \ \text{invertierbar} \land B \ \text{invertierbar} \land C \ \text{nicht invertierbar} \implies \det(C\cdot A) = 0
            $$\qed
        \end{solution}

        \part
        $\det(-A \cdot B^{-1})$
        \begin{solution}
            Wir wissen:
            $$
                \det(A\cdot B) = 1 \land \det(A) = 10 \implies \det(B) = \frac{1}{10} \implies \det(B^{-1}) = 10
            $$
            und
            $$
                \det(-A) =\footnote{$\forall M \in K^{n\times n} : \det(\lambda M) = \lambda^n \cdot \det(M)$, da jede Zeile mit $\lambda$ skaliert wird.} (-1)^2 \cdot \det(A) = \det(A)
            $$
            Damit gilt schließlich:
            $$
                A \ \text{invertierbar} \land B \ \text{invertierbar} \implies \det(-A \cdot B^{-1}) = \det(-A) \cdot \det(B^{-1}) = 100
            $$\qed
        \end{solution}
        \part
        $\det(B \cdot C - A\cdot B^{-1} \cdot C)$
        \begin{solution}
            Wir wissen, dass Matrixmultiplikation (rechts-)distributiv ist mit
            $$
                (A+B)C = AC + BC
            $$
            Damit gilt:
            $$
                \det(B \cdot C - A\cdot B^{-1} \cdot C) = \det((B - A\cdot B^{-1}) \cdot C) =\footnote{$C$ ist nicht invertierbar $\iff \abs{C} = 0$} \det(B - A\cdot B^{-1}) \cdot \det C =  0
            $$\qed
        \end{solution}
        \part
        $\det(C + C)$
        \begin{solution}
            Wir wissen:
            $$
                \det(C+C) = \det(2C) = 2^2\det(C) = 4\det(C) \land C \ \text{nicht invertierbar} \implies \det(C+C) = 0
            $$\qed
        \end{solution}

    \end{parts}
\end{questions}
\end{document}